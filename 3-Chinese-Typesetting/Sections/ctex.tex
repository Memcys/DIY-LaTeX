\CTeX 宏集手册\cite{ctex} 中写道:“\CTeX 宏集是由 \href{http://bbs.ctex.org/}{\CTeX 社区}发起并维护的 \LaTeX 宏包和文档类的集合。社区另有发布名为 \CTeX 套装的 \TeX 发行版,与本文档所述的 \CTeX 宏集并非是同一事物。“

该文档中 \S 4.1 编译方式一节写道:“\CTeX 宏集会根据用户使用的编译方式,在底层选择不同的中文支持方式“(见表 \ref{tb:ctex})。现在我们可以说,目前使用 \LaTeX 排版中文,推荐的方案是调用 xeCJK/ctex 宏包,并相应使用 \hologo{XeLaTeX} 引擎编译生成 PDF 文件。当使用 latexmk 工具(可参见附录 \ref{ap:latexmk})自动编译时,相应添加参数 `-xelatex'.

\begin{table}
\caption{\CTeX 宏集的中文支持方式}
\label{tb:ctex}
\centering
\begin{tabular}{*{5}{c}}
\toprule
编译方式 & (pdf)\LaTeX & \hologo{XeLaTeX} & \hologo{LuaLaTeX} & up\LaTeX \\ \midrule
支持宏包 & CJK & xeCJK & LuaTeX-ja & 原生 \\ \bottomrule
\end{tabular}
\end{table}

\subsection{ctex 文档类和宏包}
ctex 文档类包括 ctexart, ctexrep, ctexbook 和 ctexbeamer 分别是对标准文档类 article, report, book 和 beamer 的封装。ctex 文档类调用了 ctex 宏包。

推荐使用 ctex 文档类的情景:文档主体为中文。

一个最简单的 ctexart 文档类实例可以是如下这样的:
\texinputlst[label=lst:demo-ctex]{Demos/demo-ctex.tex}

该宏集中现提供宏包 ctex.sty, ctexsize.sty, ctexheading.sty.

\begin{Ex}{使用 ctex 宏包}{ctex-package}
请在任一英文 \TeX 文档中调用 ctex 宏包,并在正文中插入中文。编译得到 PDF 文档。
\end{Ex}

\subsection{\textbackslash ctexset\{\}}
下面简单介绍导言区设置 \verb|\ctexset{(键值列表)}| 中的一部分键值 (见 \cite{ctex} 中 \S 5.3, 6.1).
\begin{itemize}
\item scheme = (\textbf{chinese}|\emph{plain})
\begin{description}
  \item[chinese] 调整默认字号;汉化标题名字(如“表”等);在 heading = true 时,将章节标题修改为中文样式。
  \item[plain] 仅提供中文支持功能,而不对文章版式进行任何修改。 
  \end{description}
\item linespread = (数值)\\
设置行距倍数(接受浮点值)。初始值依赖于 scheme.
\item today = (\textbf{small}|\emph{big}|\emph{old})
\begin{description}
  \item[small] 效果为“2019 年 10 月 27 日”
  \item[big] 效果为“二〇一九年十月二十七日”
  \item[old] 效果为原本的(英文) 日期格式 
\end{description}
\end{itemize}

\begin{Ex}{查阅 ctex 宏集手册}{ctex}
请查阅《ctex 宏集手册》中任一感兴趣内容。
\end{Ex}