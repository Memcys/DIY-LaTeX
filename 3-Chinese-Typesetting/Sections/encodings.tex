这里对编码稍作提及。包含中文的常见编码方式有 UTF-8 和 GBK.

GBK 编码常见于 Windows 中文版操作系统。当前版本的 Win 10 系统已经支持修改默认编码 (encoding) 为 Unicode UTF-8.

至于为什么我推崇 UTF-8 而舍弃 GBK,这里我只说一点:使用 \hologo{XeLaTeX}、\hologo{LuaLaTeX}、up\LaTeX 引擎编译时, C\TeX 宏集只支持 UTF-8 编码(见文档 \cite{ctex} \S 4.1 编译方式一节)。其余方面请自行了解(如有兴趣)。

使用 UTF-8 和 \hologo{XeLaTeX}/\hologo{LuaLaTeX} 引擎时,所有 Unicode 字符均可以直接输入。例如,所有中文标点,包括单/双引号 ‘’“” 均直接输入。但仍建议英文单双引号以 \verb|`'| 或 \verb|``''| 的形式输入。

\begin{Ex}{编码转换}{encoding}
请尝试将一 GBK 编码的文件以 UTF-8 编码打开并保存。然后将新文件再次以 GBK 编码打开并保存。
\end{Ex}

\begin{notice}{强烈建议}
将源代码文件(包括但不限于 \LaTeX 相关源代码文件)以 UTF-8 编码保存。
\end{notice}