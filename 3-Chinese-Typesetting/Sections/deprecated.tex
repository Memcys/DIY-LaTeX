“CTeX 套装支持 CJK, xeCJK, CCT, TY 等多种中文 \TeX 处理方式。”
\href{http://www.ctex.org/CTeXReleaseNotes}{CTeX Release Notes} 中最新版本为 v2.9.2.164 --- 2012.03.22.

讽刺的是,其“\href{http://www.ctex.org/OldNews}{旧的新闻}”中倒数第二条声明:(2016.02.16) “新版 CTeX 套装即将发布”。

以下列举我认为已经过时的方案:
\begin{description}
\item[CTeX 套装] 一款停止维护更新的软件注定不能适应新的时代。可参见知乎专栏文章:\href{https://zhuanlan.zhihu.com/p/45174503}{[LaTeX 发行版] 2018年,为什么不推荐使用 CTeX 套装了}和\href{https://zhuanlan.zhihu.com/p/73304856}{[LaTeX 发行版] 2019 年,不用 CTEX 套装的新理由}。
\item[CJK 宏包] CTAN 上最后一个 \href{https://www.ctan.org/pkg/cjk}{CJK 宏包}发布版本为 4.8.4 2015-04-18. 而 \href{https://www.ctan.org/pkg/xecjk}{xeCJK 宏包}最后更新于 2018-05-01, 且当前仍有两位活跃的维护者 (Qing Lee, Leo Liu)。
\item[CCT 模板] 博文\href{https://liam.page/2013/10/15/LaTeX-CCT-template/}{【LaTeX Tips】国内期刊 CCT 模板编译经验}对过时的 \href{ftp://ftp.cc.ac.cn/pub/cct/}{CCT} 模板的使用有所讨论。
\item[TY] 可能过于老旧,我搜索不到相关信息。
\end{description}