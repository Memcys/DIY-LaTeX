\texinputlst[label=lst:hello]{Demos/demo-hello.tex}
没想到 helloworld 这样一款编程入门标配的程序,在要求中文输入时,却成了我四个月没能跨过去的槛。

我利用 Google, 百度搜遍关键词,得到的解决方案包括调用 xeCJK 或 ctex 宏包这样接近正确答案的搜索结果。但我那时只知道用 pdflatex 编译。我现在所读的东西似乎告诉我,Linux 操作系统下, “xeCJK/ctex 宏包 + \textbf{pdflatex} 编译”似乎不可能得到中文\footnote{但在 Windows 系统下该方案应该能够得到中文}。

我终于注意到,“ctex 宏包 + \textbf{xelatex} 编译”中的后者 xelatex. 从此,几乎\footnote{当然,我自己也写过必须使用 lualatex 编译引擎的特殊 .tex 文件。但这实在为极少数}对于任何原始英文文档,这一套解决方案几乎都能完美解决插入中文的问题。

\begin{Ex}{编译中文文档}{Chinese-helloworld}
请对 Listing \ref{lst:hello} 作必要的修改,编译得到含有中文的 PDF.

你可以先跳过,随后回头解决。也可以当即尝试解决。
\end{Ex}