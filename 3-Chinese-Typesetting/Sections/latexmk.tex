latexmk 的手册 (manual)\cite{latexmk} 中第一句话写道:`\emph{Latexmk} completely automates the process of compiling a \LaTeX document.'

以编译当前的文档 Chinese.tex 为例。可使用的编译方式为
\begin{bashlst}
latexmk -xelatex Chinese
\end{bashlst}
其中 Chinese 为 Chinese.tex 的文件名省略了后缀 `.tex'. 也可补全成 `Chinese.tex'. 实际中依次运行了 `xelatex', `biber', `xelatex' ($\times 3$),即
\begin{bashlst}[numbers=left]
xelatex Chinese
biber Chinese
xelatex Chinese
xelatex Chinese
xelatex Chinese
\end{bashlst}
值得一提的是,引擎所需的调用次数因文件而变化。

把上述 `-xelatex' 替换为 `-lualatex' 即可通过 latexmk 自动调用 lualatex (和 biber)完成相应编译。

latexmk 还有很多可选参数(调用时,只要可选参数出现 `latexmk' 之后,利用空格隔开即可,无顺序要求)。这里仅举几例:
\begin{description}
\item[-f] `Force latexmk to continue document processing despite errors.' 即无视报错,强制执行。
\item[-pv] `Run file previewer.' 即预览输出文件。
\item[-c] `Clean up (remove) all regeneratable files generated by latex and bibtex or biber except dvi, postscript and pdf.' 删除由 (pdf/xe/lua/...)latex 和 bibtex/biber 生成的中间文件,除了 dvi, ps, pdf. (有例外。)
\item[-C] `Clean up (remove) all regeneratable files generated by latex and bibtex or biber.' 删除由 (pdf/xe/lua/...)latex 和 bibtex/biber 生成的所有文件。(有例外。)
\end{description}
上述清理文件的例外可以通过配置 latexmkrc 文件解决。配合配置文件 latexmkrc, latexmk 的功能更加强大。

实际上,利用 latexmkrc 配置文件还可以配置多个文件的自动编译。例如,利用当前项目文件夹 3-Chinese-Typesetting/ 下的 .latexmkrc 文件,只需在命令行输入
\begin{bashlst}
latexmk
\end{bashlst}
随后自动编译四个文件:Demos/demo-chapter-4.tex, Demos/demo-chapter-34.tex, Chinese.tex, Chinese-beamer.tex.

其中 Chinese.tex 生成 article 文档,Chinese-beamer.tex 生成 beamer 文档。另两个 demo-*.tex 生成的 PDF 文件被 Chinese.tex 以图片形式包含。

如果只利用 latexmk 工具,至少需要
\begin{bashlst}[numbers=left]
cd Demos
xelatex demo-chapter-4
xelatex demo-chapter-34
cd ../
latexmk -xelatex Chinese
latexmk -xelatex Chinese-beamer
\end{bashlst}
而如欲删除编译产生的文件,利用 `.latexmkrc' 只需
\begin{bashlst}
latexmk -C  # 清理包括 .pdf 在内的所有生成文件
latexmk -c  # 在 '-C' 的结果中保留 .pdf 文件
\end{bashlst}

\begin{Ex}{利用 latexmk 自动编译}
请分别手动和使用 latexmk 完成 Chinese.tex 文件的编译,并尝试多种方式删除编译生成的所有文件。\\ \myhrule

提示:生成的 PDF 文档中有两个半角符号 `?'. 可以搜索自己手动编译生成的 PDF 文档中问号的数量是否多于两个初步判断编译步骤是否全部完成。此外,还可通过编译日志 Chines.log 文件及编译器的输出查询编译步骤是否欠缺。
\end{Ex}

如欲初步了解 latexmkrc, 我推荐博文\href{https://blog.prosight.me/2019/05/23/latexmk/}{如何配置 latexmk---利用latexmkrc来生成自己的latexmk}. 详尽配置信息请参阅 \href{http://personal.psu.edu/jcc8//software/latexmk-jcc/}{latexmk 手册}。

这里列出我的 `.latexmkrc' 文件
\bashinputlst[label=lst:latexmkrc]{.latexmkrc}

对于支持\emph{魔法注释} (\emph{magic comments}) 的编辑器,在主文件的\emph{首行}添加诸如
\begin{texlst}[numbers=left]
% !TeX program = latexmk -xelatex
\end{texlst}
的魔法注释也是方便的。当然,其本质是编辑器调用 latexmk 来实现。