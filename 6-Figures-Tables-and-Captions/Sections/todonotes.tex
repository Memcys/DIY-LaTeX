这里仅举例介绍 todonotes 宏包的基本用法。其它一些替代宏包有 luatodonotes\todo[fancyline]{比 todonotes 更强大,但须通过 lua(la)tex 引擎编译}, easy-todo, fixmetodonotes, todo, fixme.

\begin{vertlst}
\centering
相见欢·无言独上西楼 \\
【南唐】 李煜 \\
无言独上西楼,月如钩。寂寞梧桐深院,锁清秋。 \\
剪不断,理还乱,是离愁。别是一般\todo[inline,inlinewidth=1.7cm,noinlinepar,size=\tiny,textcolor=black!70,backgroundcolor=white,bordercolor=white]{后人也常用“番”}滋味,在心头。

\vspace{1cm}
临江仙 \\
【北宋】宴几道 \\
浅浅余寒春半,雪消蕙草初长。烟迷柳岸旧池塘。风吹梅蕊闹,雨细杏花香。 \\
月堕枝头欢意,从前虚梦高唐\todo[inline,inlinewidth=1.6cm,noinlinepar,size=\tiny,textcolor=black!70,backgroundcolor=white,bordercolor=white]{高唐:楚襄王曾梦遇神女。此喻好事如梦。},觉来何处放思量。如今不是梦,真个到伊行。
\end{vertlst}

\begin{vertlst}
\missingfigure[figwidth=8cm,figcolor=magenta!10]{Testing a long text string}
\end{vertlst}

\begin{Ex}{\LaTeX 批注}
请使用 todonotes 宏包的 \verb+todo[]{}+ 命令对一个 \LaTeX 文档做批注。
\end{Ex}