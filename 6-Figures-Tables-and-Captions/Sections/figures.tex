\subsection{插图}
正文中通过可以通过如下方式插入图片:
\begin{vertlst}
% \usepackage{graphicx} % Provide \includegraphics[]{}
% \usepackage{mwe}      % Provide images example-image and example-image-c.
  
% 行内图片
行
\includegraphics[height=2\ccwd]{example-image-c}
内
\includegraphics[height=2\ccwd,angle=45]{example-image-c}
图
\includegraphics[totalheight=2\ccwd,angle=90]{example-image-c}
片
\reflectbox{\includegraphics[totalheight=2\ccwd]{example-image-c}}
。
\end{vertlst}
或
\begin{vertlst}
% 行间图片
行间 \\
\includegraphics[scale=.5]{example-image} \\
图片。
\end{vertlst}

\verb+\includegraphics+ 有不少可选参数。常用的可能有
\begin{description}
  \item[angle] 指定旋转角度 ($^\circ$)。
  \item[width] 指定图片宽度。例如, width=5cm; 或者, width=.5\verb+\linewidth+ 等。
  \item[height] 指定图片高度。例如, height=.35\verb+\textheight+.
  \item[totalheight] 制定图片全高。当图片旋转时,效果不同于 height.
  \item[scale] 放缩。如 scale=.5 表示缩小为 0.5 倍。 
  \item[trim] 选定矩形区域。例如 trim = 1 2 3 4 选定图片区域为距左部 1bp, 距下部 2bp, 距右部 3bp, 距上部 4bp 的矩形区域。 可自行指定单位。
  \item[clip] 按选定的区域裁减。
  \item[keepaspectratio] 保持长宽比例。可设置为 true 或 false. 
  \item[page] 指定插入的 PDF 页面。 
\end{description}
如果在导言区添加
\begin{texlst}
\graphicspath{{Demos/}{Images/}}
\end{texlst}
则将在插入图片时在上述两个文件夹下搜索图片(注意 \verb+/+ 不可缺少)。


\subsection[浮动体环境]{\cnen[figure]{浮动体环境}}
在装有 LaTeX Workshop 插件的 VS Code 编辑器中,键入 `figure' 可通过回车键得到如下自动补全的模板内容:
\begin{texlst}
\begin{figure}[]
  \centering
  
  \caption{}
  \label{}
\end{figure}
\end{texlst}
在第 3 行添加
\begin{texlst}
\includegraphics[]{path/to/figure}
\end{texlst}
则可插入指定的图片。

以下是对上述代码的解释:
\begin{itemize}
  \item \verb+\begin{figure}+ 和 \verb+\end{figure}+ 是一个 figure 浮动体环境。其后的 \verb+[]+ 中为位置参数:h,t,b,p,!,H 或它们的组合(H 依赖于宏包 float 且将覆盖其余选项)。位置参数已在“盒子、浮动体”的主题报告中介绍。值得一提的是,\LaTeX 中 \verb+[]+ 内的参数均为可选参数。
  \item \verb+\centering+ 指定水平中心对齐。也可使用 center 环境 \\
\begin{texlst}[numbers=none]
\begin{center}
...
\end{center}
\end{texlst}
    代替。需要注意二者的作用域不同。
  \item \verb+\includegraphics[]{}+ 插入图片(支持格式 .png, .jpeg, .pdf, .eps, .ps, .tiff 等)
  \item \verb+\caption+ 添加图注。
  \item \verb+\label{}+ 添加标签。可在文中通过 \verb+\ref{}+ 引用。需注意,\verb+\label{}+ 必须出现在 \verb+\caption{}+ 后。
\end{itemize}

此外,还可以通过 float 宏包给图片添加边框。
\texinputlst{Demos/demo-boxed.tex}
效果如图 \ref{demo-boxed}.
\begin{figure}[]
  \centering
  \includegraphics[trim=4.6cm 11cm 4.6cm 11cm, clip]{demo-boxed}
  \caption{demo-boxed}
  \label{demo-boxed}
\end{figure}