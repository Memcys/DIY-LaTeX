利用 \href{http://www.ctan.org/pkg/subcaption}{subcaption 宏包} \cite{subcaption},在 figure 环境中嵌入 subfigure 环境插入子图。
\texinputlst{Demos/demo-subcaption.tex}
\begin{figure}[]
  \centering
  \includegraphics[trim=4.4cm 11.8cm 4.4cm 12.cm, clip]{demo-subcaption}
  \caption{subcaption 宏包示例}
  \label{demo-subcaption}
\end{figure}
同样地,在 table 环境中嵌入 subtable 环境插入子表格。以下为用法示例:
\begin{texlst}
\begin{table}[<placement specifier>]
  \begin{subtable}[<placement specifier>]{<width>}
      \centering
      ... table 1 ...
  \caption{<sub caption>}
  \end{subtable}
  ~
  \begin{subtable}[<placement specifier>]{<width>}
      \centering
      ... table 2 ...
      \caption{<sub caption>}
  \end{subtable}
\end{table}
\end{texlst}

如果需要将图拆分成多个部分,可通过以下代码实现图 \ref{demo-continuedfloat} 的效果。
\texinputlst{Demos/demo-continuedfloat.tex}
\begin{figure}[]
  \centering
  \includegraphics[trim=7.5cm 21.cm 7.cm 3.8cm, clip]{demo-continuedfloat}
  \caption{ContinuedFloat 示例}
  \label{demo-continuedfloat}
\end{figure}

\begin{Ex}{插入含子图的浮动体}
请尝试插入一个含有多个子图/子表的浮动体,并分别给出标注。
\end{Ex}