\subsection{tabular 环境}
tabular 环境是我常用的表格环境。它本身不是浮动体。示例如下:
\begin{sidelst}
\begin{tabular}{|r|c|l|}
\cline{1-1} \cline{3-3}
1 & & 321 \\ \hline
12 & 2 \\ \hline
123 & 123 & 3 \\ \hline
\end{tabular}
\end{sidelst}
另有 tabu, tableau 等表格环境。这里不作介绍。

表格环境多且复杂。这里将仅仅列举。希望大家自己参阅更多文档;并且调整代码,根据输出效果能够有所领会。以下主要参考 \cite{wiki:tables} 中 \href{https://en.wikibooks.org/wiki/LaTeX/Tables\#Floating\_with\_table}{Floating with table} 一节。
\begin{sidelst}
\centering     % center the following objects
\begin{tabular}{r | l c}  % alignment: {r: right; l: left; c: center}
1 & 2 & 3 \\              % &: new column; \\: new row
4 & 5 & 6 \\
7 & 8 & 9 \\ \hline       % hline: horizontal line
\end{tabular}
\\[1cm]
\begin{tabular}{|r|l|}    % | vertical line
\hline
7C0 & hexadecimal \\
3700 & octal \\ \cline{2-2} % horizontal line from i to j
11111000000 & binary \\
\hline \hline
1984 & decimal \\ \hline
\end{tabular}
\\[1cm]
\begin{tabular}{cc}
\hline % \parbox: paragraph box; automatically wrap the paragraph
cell content & \parbox[t]{2.cm}{rather long par\\new par}
\\ \hline
\end{tabular}
\\[1cm]
\begin{tabular}{r@{.}l} % @{} seperation between columns
\hline
3   & 14159 \\
16  & 2     \\
123 & 456   \\ \hline
\end{tabular}
\end{sidelst}

\begin{vertlst}
% \usepackage{dcolumn}
\newcolumntype{d}[1]{D{.}{\cdot}{#1} }
%the argument for d specifies the maximum number of decimal places
\begin{tabular}{l r c d{1} }
\toprule
L & R & C & \multicolumn{1}{c}{Decimal} \\ \midrule
1 & 2 & 3 & 4 \\
11 & 22 & 33 & 44 \\
1.1 & 2.2 & 3.3 & 4.4 \\ \bottomrule
\end{tabular}
\\[.5cm]
\begin{tabular}{ |l|l|l| }
\hline
\multicolumn{3}{ |c| }{Team sheet} \\ \hline
Goalkeeper & GK & Paul Robinson \\ \hline
\multirow{4}{*}{Defenders} & LB & Lucas Radebe \\
  & DC & Michael Duburry \\
  & DC & Dominic Matteo \\
  & RB & Didier Domi \\ \hline
\end{tabular}
\\[.5cm]
\begin{tabular}{ *{3}{| c} | p{4.5cm} |}
\hline
Day & Min Temp & Max Temp & Summary \\ \hline
Monday & 11C & 22C & A clear day with lots of sunshine. \\ \hline
\end{tabular}
\end{vertlst}
这里我推荐阅读 \cite{texblog-table}.

使用 table 环境就实现了一个类似 figure 环境的浮动体。如果在 table 中嵌入 tabular 环境,则用法如:
\begin{texlst}
\begin{table}
  \caption{}
  \label{}
  \centering
  \begin{tabular}{}
    
  \end{tabular}
\end{table}
\end{texlst}

\subsection{长表格}
有的图表等可能过长或过宽。可以考虑缩小字体大小、列间距,横向 (landscape) 放置或跨页放置。

横向放置可使用 landscape 环境。多个宏包 (geometry, lscape, pdflscape) 提供了 landscape 相关选项。甚至可以直接在文档类选项中指定 landscape. 请参见 \cite{texblog-landscape}. 这里以 pdflscape 宏包为例。效果见表 \ref{demo-pdflscape} 所在页面。
\texinputlst{Demos/demo-pdflscape.tex}

\begin{landscape}
\begin{table}
\caption{山海经·南山经} \label{demo-pdflscape}
\centering
\begin{tabular}{b{5.2cm} m{5.2cm} p{5.2cm}}
\toprule
\zhlipsum[2][name=nanshanjing] & \zhlipsum[2][name=nanshanjing] & \zhlipsum[2][name=nanshanjing] \\ \midrule
\multicolumn{3}{l}{\zhlipsum[3][name=nanshanjing]} \\ \bottomrule
\end{tabular}
\end{table}
\end{landscape}

宏包 longtable 可以产生跨页的表格。具体请参见其\href{http://mirrors.ctan.org/macros/latex/required/tools/longtable.pdf}{宏包文档} \cite{longtable}. 根据 \cite{wiki:tables} \href{https://en.wikibooks.org/wiki/LaTeX/Tables#The\_tabu\_environment}{Table across several pages} 一小节, tabu 宏包提供的 longtabu 环境也可以跨页。


\subsection[CSV to LaTeX tables]{由 CSV 文件生成 \LaTeX 表格}
CSV 是 Comma Separated Value 的缩写,中文也称“逗号分隔值”。以下参考 \cite{texblog-csv}. 其中讨论了宏包 csvsimple, pgfplotstable. 这里仅以宏包 csvsimple 为例。

假设我们有一个待插入的 CSV 文件:
\bashinputlst{Demos/scientists.csv}
我们可以通过以下代码实现自动插入:
\begin{sidelst}
% \usepackage{csvsimple}

\csvautotabular{Demos/scientists.csv}
\end{sidelst}

另外,正如“盒子、浮动体”报告中提到的,也可以通过 \href{https://github.com/krlmlr/Excel2LaTeX}{Excel2LaTeX}\todo[fancyline]{Office Excel 插件}, \href{http://calc2latex.sourceforge.net/}{Calc2LaTeX}\todo{Libre Office Calc 插件} 等插件从其他表格数据文件导出 \LaTeX 格式的表格代码。

\begin{Ex}{实现 \LaTeX 表格}
请自己实现一个 \LaTeX 表格,无论是手写,从 CSV 导入,还是由其他插件生成。
\end{Ex}