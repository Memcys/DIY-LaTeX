\documentclass[hyperref,UTF8]{ctexart}

\usepackage{geometry}
\usepackage{graphicx}
\usepackage{float}
\usepackage[format=hang,font=small,textfont=it]{caption}%改变图表标题格式的caption宏包,设定所有图表标题格式使用悬挂对齐方式(即编号向左突出),整体使用小字号,而标题文本使用斜体(对汉字来说是楷书)
\usepackage{amsmath}
\usepackage{amsfonts,amssymb,bm,amsthm}
\usepackage{esint}%二重环路积分
\usepackage{cancel}%\cancel,\xcancel,\bcancel删除线
\usepackage{mathtools}%\prescript加左上下标
\usepackage{cases}%大括号编号
%\usepackage{ulem}%使用改良的下划线\uline
\usepackage{mathrsfs}%\mathscr使用花写的L
\usepackage{upgreek}%直立体希腊字母
\usepackage{enumitem}%输出一、二、三、
\usepackage{mhchem}%\ce输入化学式
\usepackage{hyperref}
\usepackage{listings}
\usepackage{color}
\usepackage{fontspec}
\usepackage{xltxtra}%提供XeLaTeX标识符
\usepackage{texnames}%提供AMS标识符
\usepackage[nottoc]{tocbibind}%增加目录的项目的tocbibind宏包,默认在目录中加入目录项本身、参考文献、索引等项目。使用nottoc取消目录中显示目录本身
\usepackage{booktabs}
\usepackage{longtable}

% \setmonofont{Consolas}

\definecolor{codegreen}{rgb}{0, 0.6, 0}
\definecolor{codegray}{rgb}{0.5, 0.5, 0.5}
\definecolor{codepurple}{rgb}{0.58, 0, 0.82}
\definecolor{backcolor}{rgb}{0.95, 0.95, 0.92}
%\definecolor{codered}{rgb}{0.647, 0.164, 0.164}
\definecolor{codered}{rgb}{0.546875, 0, 0}
\newcommand{\red}[1]{\textcolor{red}{#1}}
\newcommand{\codegreen}[1]{\textcolor{codegreen}{#1}}
\newcommand{\blue}[1]{\textcolor{blue}{#1}}

\lstset{
	language=[AlLaTeX]TeX,
%	frame=tb,
	aboveskip=3mm,
	belowskip=3mm,
	morekeywords={subseteqq, coloneq, coloneqq, geometry, maketitle, zihao, command,leqslant, geqslant, XeLaTeX, AmS, AmSTeX, BibTeX, LuaTeX, CTeX, CCTeX, mathbb, prescript, indices, tensor, ce, underrightarrow, underleftarrow, overleftrightarrow, underleftrightarrow, text, overbracket, underbracket, dfrac, tfrac, cfrac, binom, dbinom, tbinom, mathnormal, mathcal, mathscr, mathfrak, varkappa, digamma, varGamma, varDelta, varTheta, varLambda, varXi, varPi, varSigma, varUpsilon, varPhi, varPsi, varOmega, upalpha, upbeta, upgamma, updelta, upepsilon, upzeta, upeta, uptheta, upiota, upkappa, uplambda, upmu, upnu, upxi, uppi, uprho, upsigma, uptau, upupsilon, upphi, upchi, uppsi, upomega, upvarepsilon, upvartheta, upvarpi, upvarrho, upvarsigma, upvarphi, dddot, ddddot, mathring, hslash, varnothing, square, blacksquare, lozenge, blacklozenge, bigstar, complement, mho, iint, iiint, idotsint, oiint, varoiint, iiiint},
	backgroundcolor=\color{backcolor},   
	commentstyle=\color{red},
	keywordstyle={\color{blue} },
%	morekeywords={coloneq}
%	basicstyle={\small\ttfamily},
	basicstyle={\zihao{5}\ttfamily},
	showstringspaces=false,
	columns=flexible,
	breaklines=true,
	breakatwhitespace=true,
	tabsize=4,
%	captionpos=b,
	keepspaces=true,
	numbers=left,                  
	numbersep=5pt,
	numberstyle=\tiny\color{codegray},
	showspaces=false,
	showstringspaces=false,
	showtabs=false,
	xleftmargin=8pt,
	escapechar="
}

\hypersetup{
	colorlinks=false,
	bookmarks=true,
	bookmarksopen=ture,
	bookmarksnumbered=ture,
	pdfborder=0 0 1,
%	pdfpagemode=FullScreen,
	pdfstartview=FitH,
	pdftitle={广义相对论讲义},
	pdfauthor={黄超光、樊兆兴},
	pdfsubject={广义相对论},
	pdfkeywords={黑洞理论、宇宙学、微分几何与张量分析、引力波、引力场方程}
}

\makeatletter
\@addtoreset{equation}{section}
\makeatother%使section后公式编号清零

\title{\heiti 数学模式与 \AmS{} 宏集}
\author{\kaishu 梁昊 \\ \kaishu 樊兆兴}
\date{\today}

\geometry{a4paper,centering,left=2cm,right=2cm,top=2cm,bottom=2cm}%scale=0.8
\linespread{1.5}
\numberwithin{equation}{section}
\allowdisplaybreaks[4]

\newtheorem{thm}{定理}[section]
\newtheorem{lemma}{引理}%[section]

\newenvironment{prove}{\noindent\textbf{证明:\quad}\kaishu}{\hfill$\qed$\par}

\bibliographystyle{unsrt}%按照引用的先后顺序排列

%定义新命令
\newcommand{\degree}{^\circ}
\newcommand{\units}[1]{\,\mathrm{#1}}
%\newcommand{\diff}[1]{\,\mathrm d #1\!}
\newcommand{\diff}[1]{\,\mathrm d #1}
\newcommand{\md}{\mathrm d}
\newcommand{\parti}[3]{\frac{\partial^{#3} #1}{\partial {#2}^{#3} } }
\newcommand{\dparti}[3]{\dfrac{\partial^{#3} #1}{\partial {#2}^{#3} } }
\newcommand{\partii}[3]{\frac{\partial^{2} #1}{\partial #2 \partial #3} }
\newcommand{\dpartii}[3]{\dfrac{\partial^{2} #1}{\partial #2 \partial #3} }
\newcommand{\derive}[3]{\frac{\mathrm{d}^{#3} #1}{\mathrm{d} {#2}^{#3} } }
\newcommand{\dderive}[3]{\dfrac{\mathrm{d}^{#3} #1}{\mathrm{d} {#2}^{#3} } }
\newcommand{\commu}[2]{\left[ #1, #2 \right]}
\newcommand{\norm}[1]{\left\| #1 \right\|}
\newcommand{\abs}[1]{\left| #1 \right|}
\newcommand{\mc}{\mathrm{c}}
\newcommand{\ms}{\mathrm{s}}
\newcommand{\diag}[1]{\operatorname{diag} \left( #1 \right)}
\newcommand{\CCTeX}{$\mathbb{C}$\kern-.05em\TeX}
\newcommand{\tensor}[3]{ {#1}_{\phantom{#3} #2}^{#3} }
%定义数集符号
\newcommand{\real}{\mathbb R}
\newcommand{\ration}{\mathbb Q}
\newcommand{\complex}{\mathbb C}
\newcommand{\integer}{\mathbb Z}
\newcommand{\natura}{\mathbb N}
%狄拉克符号
\newcommand{\braket}[1]{\left< #1 \right>}
\newcommand{\bralket}[2]{\left< #1 \middle| #2 \right>}
\newcommand{\braOket}[3]{\left< #1 \middle| #2 \middle| #3 \right>}
\newcommand{\bra}[1]{\left< #1 \right|}
\newcommand{\ket}[1]{\left| #1 \right>}
\newcommand{\inn}[2]{\left< #1, #2 \right>}
%自然常数
\newcommand{\Boltz}{k_{\mathrm{B} } }
\newcommand{\me}[1]{\mathrm{e}^{#1}}
\newcommand{\mi}{\mathrm{i}}
\newcommand{\mh}{\mathrm{h}}
%定义几种矩阵群符号
\newcommand{\Mn}[2]{M_{#1} \left( #2 \right)}
\newcommand{\GL}[2]{\operatorname{GL}_{#1} \left( #2 \right)}
\newcommand{\SL}[2]{\operatorname{SL}_{#1} \left( #2 \right)}
\newcommand{\Ortho}[1]{\operatorname{O} \left( #1 \right)}
\newcommand{\SO}[1]{\operatorname{SO} \left( #1 \right)}
\newcommand{\Uni}[1]{\operatorname{U} \left( #1 \right)}
\newcommand{\SU}[1]{\operatorname{SU} \left( #1 \right)}

\begin{document}

\begin{figure}[t]%浮动体可以出现在环境周围的文本所在处(here)和第一页的顶部(top)
	\centering%用声明\centering表示后面的内容居中
	\includegraphics[width=\linewidth]{ucaslogo.pdf}%scale为放缩因子
	%\caption命令给插图加上自动编号和标题
	%\caption{}
	%\label{fig:xiantu}%用\lable命令给图形定义一个标签,文章其它地方可以引用\caption产生的编号
\end{figure}

\maketitle

\begin{abstract}
。
\end{abstract}

\thispagestyle{empty}

\newpage

\pagenumbering{roman}
\tableofcontents%tableofcontents命令输出目录

\newpage

\pagenumbering{arabic}
\zihao{-4}
%\kaishu

\section{\LaTeX{}的概述}

在开始介绍数学环境排版之前,很有必要大概地介绍 \LaTeX{} 的基本内容。介绍仅仅是形式上的,不涉及\LaTeX{} 和 \TeX{} 之间复杂的原理。这篇文章的内容全部由 \XeLaTeX{} 引擎和 \BibTeX{} 引擎编译得到。

\subsection{\LaTeX{}代码结构归纳}\label{subsec:结构}

在本人的码字经验看来,源文件的结构一般是:
\begin{lstlisting}
\documentclass[hyperref,UTF8]{ctexart}%"\color{red} 声明文档类(documentclass)"

%"\color{red} 调用的宏包(package)"
\usepackage{geometry}
\usepackage{amsmath}
\usepackage{amsfonts, amssymb, amsthm}
\usepackage{bm}
\usepackage{mathtools}
\usepackage{mathrsfs}
\usepackage{upgreek}
"$\dots\dots$"

\title{"这次的讲义"}
\author{D.E.D \and "芋圆公式"}
\date{\today}

%"\color{red} 定义页面使用A4纸大小,版心居中,四个方向缩进2厘米"
\geometry{a4paper, centering, left=2cm, right=2cm, top=2cm, bottom=2cm}
\linespread{1.5}%"\color{red} 设置行距为1.5倍"
"$\dots\dots$"

\begin{document}

\maketitle%"\color{red} 产生标题"

\zihao{-4}%"\color{red} 正文内容开始使用小四字号"
"$\dots\dots$"

\end{document}%"\color{red} \verb!\end{document}! 后的内容都不参与编译。"
\end{lstlisting}
首先暂时无视这些命令具体的意义,随即将要介绍命令的基本构成。\lstinline!\begin{document}! 和 \lstinline!\end{document}! 开始和结束的是 \verb!document! 环境(22--29行),是直接输出的正文。\verb!document! 环境之前的内容(1--21行)称为\textbf{\songti 导言区}(preamble),用来设置文档的性质和自定义命令。\verb!document! 环境之后的所有内容都会被 \LaTeX{} 忽略。

导言区第一行 \lstinline!\documentclass[hyperref,UTF8]{ctexart}! 使用了文档类 \verb?ctexart? ,并用选项 \verb-[UTF8]- 说明了编码。4--11行为调用的宏包(package),列出的宏包支持了文章里所有数学排版。13--20行设置了文档的全局性质,例如 \lstinline'\title{}'、 \lstinline'\author{}' 和 \lstinline'\date{}' 声明了文章的标题、作者和写作日期,这些信息不会马上出现在编译得到的PDF中;\lstinline'\geometry{}' 设定了页面大小;\lstinline'\linespread{}' 设置了行距。

在导言区声明过的文章的标题、作者和写作日期需要通过 \lstinline'\maketitle' 输出到PDF中。\lstinline!\zihao{-4}! 声明这之后的正文内容都采用小四字号。正文部分还有更多复杂的章节划分和文本环境等待后面的同学讲解。

\subsection{命令}

\LaTeX{} 中的命令全部以反斜杠(``\lstinline!\!", backslash)开头,因此``\lstinline!\!"被称为\textbf{\songti 命令前导符},左端带有``\lstinline!\!"的字符串都被认为是命令。命令的形式无非
\begin{enumerate}[label=(\arabic*)]

\item ``\lstinline!\!"紧跟一串大、小写英文字母。例如:\lstinline!\documentclass[hyperref,UTF8]{ctexart}! 、\lstinline!\zihao{-4}! 以及 \lstinline!\maketitle!。

\item ``\lstinline!\!"紧跟一个非字母符号。例如:\lstinline!\!\textvisiblespace
\protect\footnote{
	本人尝试解释清楚所有的内容:表示命令前导符紧跟一个空格,它产生的效果也确实是输出一个空格。
}
、\lstinline!\%!
\protect\footnote{
	输出注释符,也就是百分号。
}和 \lstinline!\\!
\protect\footnote{
	并不会输出命令前导符反斜杠,这里整体表示一个换行符。
}
。

\end{enumerate}
这两种。很容易归纳得到命令具有以下的基本形式:
\begin{lstlisting}[numbers=none]
	\command["可选参数1", "可选参数2"...]{"必要参数1"}...{"必要参数$n$"}
\end{lstlisting}
,即命令前导符后紧跟命令名称、可选参数列表和$n$个必要参数。
\begin{enumerate}[label=(\arabic*)]

\item $n$个必要参数分别由 \lstinline!{}! 包括,其中$n = 0, 1, 2, \dots$。必要参数如果缺少,编译时会报错。

\item 可选参数列表由 \lstinline![]! 包括,可选参数的个数不定,可以是0。多个可选参数彼此用 \lstinline!,! 分隔。例如:\lstinline!\documentclass[hyperref,UTF8]{ctexart}!。

\end{enumerate}

\subsection{文档类}

\LaTeXe{} 基本的文档类有 \verb'article'、 \verb'report' 和 \verb'book' 三个,分别用来编写小篇幅的文章、中篇幅的报告和长篇幅的书籍。\CCTeX{} 组织编写了 \verb-ctex- 文档类,是 \LaTeXe{} 基本文档类的中文对应物。\verb-ctex- 宏包提供三个文档类有 \verb'ctexart'、 \verb'ctexrep' 和 \verb'ctexbook' 分别对应上述三个文档类,用来编写中文短文、中文报告和中文书籍。另外还有 \verb'beamer' 文档类用以制作论文陈述演示文稿,例如这次的演示文稿。
\footnote{
	一点补充内容:中文 \LaTeX{} 常用 \CCTeX{} 套装——由中国科学院数学与系统科学研究院的吴凌云研究员维护。中国科学院大学学位论文 \LaTeX{} 模板(\LaTeX{} Thesis Template for the University of Chinese Academy of Sciences)得到了国科大本科部陆晴老师、本科部学位办丁云云老师和中科院数学与系统科学研究院吴凌云研究员的支持。这一模板基于中科院数学与系统科学研究院吴凌云研究员的CASthesis模板发展而来。\href{https://github.com/mohuangrui/ucasthesis}{https://github.com/mohuangrui/ucasthesis}
}

\subsection{宏包}

基础 \TeX{} 命令和 \LaTeX{} 命令组合为宏(macro),宏是一种抽象,它根据一系列预定义的规则替换一定的文本模式。宏包储存这些宏。宏包按照以下格式在导言区调用:
\begin{lstlisting}[numbers=none]
	\usepackage["可选参数1", "可选参数2"...]{"宏包名称"}
\end{lstlisting}
。例如,调用宏包 \verb'amsmath',则命令 \lstinline!\leqslant! 和 \lstinline!\geqslant! 已经定义并可以使用,它们分别输出$\leqslant$和$\geqslant$。再比如 \lstinline!\usepackage[nottoc]{tocbibind}! 调用了 \verb'tocbibind' 宏包增加目录的项目,默认在目录中加入目录项本身、参考文献、索引等项目。使用选项 \lstinline![nottoc]! 取消目录中显示目录本身。

\ref{subsec:结构} 小节中导言区有两种调用宏包的方式:
\begin{lstlisting}
	\usepackage{amsmath}
	\usepackage{amsfonts, amssymb, amsthm}
\end{lstlisting}
。显然,需要设置可选参数的宏包需要采用第1行的调用方式,其它默认选项的宏包可以采用第2行的方式统一调用。

\begin{table}[H]
\centering
\begin{tabular}{cl|cl}
	\toprule
	名称 & \multicolumn{1}{c|}{作用简介} & 名称 & \multicolumn{1}{c}{作用简介} \\
	\midrule
	\verb'amsmath' & 支持大量数学公式环境和数学符号 & \verb'amsfonts' & 支持了黑板粗体(例如:$\mathbb{R}$)等 \\
	\verb'amssymb' & 支持哥特体(例如:$\mathfrak{S}$)等 & \verb'amsthm' & 定制定理类环境 \\
	\verb'bm' & 输出粗斜体,例如:$\bm{a} \times \bm{b} = \bm{0}$ & \verb'geometry' & 页面尺寸的设置 \\
	\verb'graphicx' & 处理插图的各种格式 & \verb'float' & 提供容纳插图表格的浮动体 \\
	\verb'caption' & 改变图表的标题格式 & \verb'color' & 自定义和预定义各种颜色 \\
	\verb'esint' & 支持美观的二重环路积分符号等 & \verb'cancel' & 产生各种取向的删除线 \\
	\verb'mathtools' & 排版复杂的上下标,例如张量 & \verb'mathrsfs' & 支持数学花体,例如:$\mathscr{F}$ \\
	\verb'upgreek' & 独立地支持直体希腊字母输出 & \verb'enumitem' & 定制列表环境,产生中文序号 \\
	\verb'hyperref' & 产生超链接和PDF书签 & \verb'listings' & 改良的抄录代码环境 \\
	\verb'xltxtra' & 产生标识符 \XeLaTeX & \verb'texnames' & 产生多种标识符,例如 \AmS{} \\
	\verb'tocbibind' & 增加目录的项目 & \verb'booktabs' & 绘制三线表并控制表线 \\
	\bottomrule
\end{tabular}
\caption{常用宏包和作用列表}
\label{table:常用宏包}
\end{table}
本人平常经常使用的宏包如表 \ref{table:常用宏包} 所示。

\subsection{模式}

模式(mode)是处理源文件的方式,有左右模式(left-to-right mode, LR mode)、段落模式(paragraph mode)和数学模式(math mode)。左右模式中的内容不换行,比如在各种盒子(box)中的内容。常规的文本默认遵循段落模式排版,系统自动分行、分段和分页。数学模式将是 \ref{sec:AMS} 节的重点。

\subsection{符号}

符号分为专用符号、文本符号和数学符号。

专用符号是指10个被 \LaTeX{} 赋予了特殊用途的符号,它们的用途和名称如表 \ref{table:专用符号} 所示。
\begin{table}[H]
\centering
\begin{tabular}{ccc|l}
	\toprule
	符号 & 名称 & 在文本中的输出方式 & \multicolumn{1}{c}{用途} \\
	\midrule
	\lstinline'%' & 注释符 & \lstinline'\%' & 注释符之后的内容都在编译时被忽略 \\
	\lstinline'\' & 命令前导符 & \lstinline'\textbackslash' & 开启一个 \LaTeX{} 命令 \\
	\lstinline'{}' & brace & \lstinline'\{\}' & 标志必要参数或组合 \\
	\lstinline'$' & 数学模式符 & \lstinline'\$' \lstinline'\textdollar' & 成对的 \lstinline'$' 标志了数学模式 \\
	\lstinline'&' & 分列符 & \lstinline'\&' & 在各种表格环境中表示列之间的分隔 \\
	\lstinline'^' & 上标符 & \lstinline'\^{}' \lstinline'\textasciicircum' & 在数学模式中将 \lstinline'^' 后的参数变为上标 \\
	\lstinline'_' & 下标符 & \lstinline'\_' & 在数学模式中将 \lstinline'_' 后的参数变为下标 \\
	\lstinline'#' & 参数符 & \lstinline'\#' & 自定义命令时标志参数 \\
	\lstinline'~' & 空格符 & \lstinline'\~{}' \lstinline'\textasciitilde' & 产生一个不可换行的空格 \\
	\bottomrule
\end{tabular}
\caption{专用符号的名称、在文本中的输出方式和用途}
\label{table:专用符号}
\end{table}
\noindent 命令前导符的输出方式与众不同的原因是显然的。brace亦可通过命令 \lstinline'\textbraceleft' 和 \lstinline'\textbraceright' 输出。\lstinline'|'、\lstinline'<' 和 \lstinline'>' 这三个符号被直接定义为数学符号,只能出现在数学模式中,在文本中可以用 \lstinline'\textbar'、\lstinline'\textless' 和 \lstinline'\textgreater' 分别输出。

\begin{table}[H]
\centering
\begin{tabular}{ccc|ccc}
	\toprule
	符号 & 生成命令 & 所需调用的宏包 & 符号 & 生成命令 & 所需调用的宏包 \\
	\midrule
	\TeX & \lstinline'\TeX' & ~ & \LaTeX & \lstinline'\LaTeX' & ~ \\
	\LaTeXe & \lstinline'\LaTeXe' & ~ & \XeLaTeX & \lstinline'\XeLaTeX' & \verb'xltxtra' \\
	\AmS & \lstinline'\AmS' & \verb'texnames' & \AmSTeX & \lstinline'\AmSTeX' & \verb'texnames' \\
	\BibTeX & \lstinline'\BibTeX' & \verb'texnames' & \LuaTeX & \lstinline'\LuaTeX' & \verb'texnames' \\
	\CTeX & \lstinline'\CTeX' & \verb'ctex' & \CCTeX & \lstinline'\CCTeX' & 自定义命令 \\
	\bottomrule
\end{tabular}
\caption{\TeX{} 家族及相关的标识符}
\label{table:TeX}
\end{table}
文本符号仅仅介绍 \TeX{} 家族及相关的标识符,如表 \ref{table:TeX} 所示。表 \ref{table:TeX} 中\lstinline!\CCTeX! 命令这样定义\cite{CTeX宏集手册}:
\begin{lstlisting}[numbers=none]
	\newcommand{\CCTeX}{$\mathbb{C}$\kern-.05em\TeX}
\end{lstlisting}
。考虑到大多数文档中 \lstinline!\mathbb{}! 未必有定义,就不采用 \CCTeX{} 了,只定义最简单的形式 \CTeX{} 。

\LaTeX{} 的概述还应包含距离、盒子、计数器、交叉引用、环境、四则运算、条件判断、注释与提示和颜色。这些使得 \LaTeX{} 成为了Turing完备的语言。

\section{\texorpdfstring{\AmS{} 宏集}{AMS宏集}}\label{sec:AMS}

现在来到 \LaTeX{} 的“顶尖级成就”——数学排版。

\verb'amsmath' 宏包由美国数学会(American Mathematical Society, AMS)设计开发,连同 \verb'amsthm' 等宏包和美国数学会文档类一起构成 \AmS-\LaTeX{} 套件,本人认为这就是 \AmS{} 宏集。

\AmS{} 涉及到数学模式的方方面面,考虑到 \LaTeX{} 知识的系统性,最好还是通过系统地介绍数学模式来探究 \AmS{} 宏集。

\subsection{简单的数学环境}

在介绍 \AmS{} 宏集之前,需要一些基本的数学环境来实现这些内容。详细的数学环境将在下周日(2019.09.29)介绍。

\TeX{} 有两种数学公式环境:
\begin{enumerate}[label=(\arabic*)]

\item \textbf{\songti 行内公式} (inline math),夹杂在行文段落之中。例如:Euler公式$\me{\mi \uppi} + 1 = 0$。

\item \textbf{\songti 行间公式} (display math),单独占据一行,与上下文有一定间距。例如:在$0 < p < 1$时,成立余元公式(the complement formula)
\begin{equation}
	\mathrm{B} \left( p, 1 - p \right) = \Gamma \left( p \right) \Gamma \left( 1 - p \right) = \frac{\uppi}{\sin p \uppi}.
\end{equation}
这里的大写希腊字母$\mathrm{B}$和$\Gamma$属于特殊函数的函数名符号,根据排版规范使用了直立罗马体。

\end{enumerate}

\subsubsection{行内公式}

\LaTeX{} 提供了三种行内公式环境:
\begin{enumerate}[label=(\roman*)]

\item \lstinline'$ ... $';

\item \lstinline'\( ... \)';

\item \lstinline'\begin{math} ... \end{math}'。

\end{enumerate}
编译结果是完全相同的,例如:
\begin{lstlisting}
	$$a^{2} + b^{2} \geqslant 2 a b$$
	\(
		a^{2} + b^{2} \geqslant 2 a b
	\)
	\begin{math}
		a^{2} + b^{2} \geqslant 2 a b
	\end{math}
\end{lstlisting}
均会得到基本不等式$a^{2} + b^{2} \geqslant 2 a b$。但是本人强烈建议使用 \lstinline'$ ... $' 环境,不仅因为简便,还因为它不像另外两种环境是\textbf{\songti “脆弱”命令} (fragile command),在特定条件下会出错。

\subsubsection{行间公式}

\LaTeX{} 提供了三种行间公式环境:
\begin{enumerate}[label=(\roman*)]

\item \lstinline'$$ ... $$';

\item \lstinline'\[ ... \]';

\item \lstinline'\begin{displaymath} ... \end{displaymath}'。

\end{enumerate}
编译结果是完全相同的,例如:
\begin{lstlisting}
	$$ x_{n + 1} = \frac{1}{2} \left( x_{n} + \frac{a}{x_{n}} \right). $$
	\[
		x_{n + 1} = \frac{1}{2} \left( x_{n} + \frac{a}{x_{n}} \right).
	\]
	\begin{displaymath}
		x_{n + 1} = \frac{1}{2} \left( x_{n} + \frac{a}{x_{n}} \right).
	\end{displaymath}
\end{lstlisting}
均会得到
\[
	x_{n + 1} = \frac{1}{2} \left( x_{n} + \frac{a}{x_{n}} \right).
\]
\lstinline'$$ ... $$' 会产生不良的间距,\lstinline'\[ ... \]' 也同为“脆弱”命令,因此最好使用 \verb'displaymath' 环境。

然而考虑到 \LaTeX{} 提供了自动编号的单行行间公式环境 \verb'equation',而 \verb'amsmath' 宏包提供了去掉编号的 \verb'equation*' 环境,是比 \verb'displaymath' 环境更好的选择。

\subsection{基本数学结构}

纸面上数学公式复杂的结构是如何通过线性的字符串表达的呢?

\subsubsection{上标与下标}

这些绿色与红色的部分都可以归纳为上标与下标:
\begin{equation*}
	\begin{aligned}
		90^{\codegreen{\circ}} && x_{\red{1}} && y^{\codegreen{\prime}} && z^{\color{codegreen} 2} && \mathrm{C}_{\red{n}}^{\codegreen{m}} && \sup_{\red{x \in I}} && \int_{\red{0}}^{\codegreen{1}} && \int_{\red{D}} && \oint\limits_{\red{|z| = 1}} && \prod_{\red{k = 1}}^{\codegreen{n}} && \sum_{\red{1 \leqslant i < j \leqslant m}} && \tensor{T}{\red{\nu_{1} \nu_{2} \cdots \nu_{l}}}{\codegreen{\mu_{\red{1}} \mu_{\red{2}} \cdots \mu_{\red{k}}}}
	\end{aligned}
\end{equation*}
可以看到上标可以在正上方或者右上方、下标可以在正下方或者右下方。

上标和下标分别利用 \lstinline'^{}' 和 \lstinline'_{}' 输出,强烈建议将 \lstinline'^' 和 \lstinline'_' 看作带有一个必要参数的命令,可以避免很多麻烦,而且可以发现上标和下标的相互嵌套也变得容易理解了,比如命令 \lstinline'2^{n_{i}}' 输出$2^{n_{i}}$。上下标可以同时存在,可能在代码上涉及顺序问题,但是可以看到命令 \lstinline'x_{1}^{2}' 和 \lstinline'x^{2}_{1}' 都可以输出$x^{2}_{1}$,由于多种考虑建议使用前者。

其中一种关系不大的考虑是特殊的上标 \lstinline!'! 的存在
\footnote{
	英文的引号`和'分别位于标准键盘的tab键上方和enter键左方。
}
。举一个例子来说明 \lstinline!'! 的特殊:\lstinline!y_{0}'! 和 \lstinline!y_{0}^{\prime}! 都可以输出$y_{0}^{\prime}$,\lstinline!y_{0}''! 和 \lstinline!y_{0}^{\prime \prime}! 都可以输出$y_{0}^{\prime \prime}$。\lstinline!'! 可以连续使用,而普通上标连续使用会收到系统报错:\verb'! Double superscript.'
\footnote{
	报错命令的具体形式依赖于编辑器,这里的编辑器是 \TeX works。
}
,普通上标和 \lstinline!'! 的混用亦是如此。因此若是需要${s'}^{2}$的排版效果,需要分组 \lstinline!{}! 的介入,也就是代码 \lstinline!{s'}^{2}!。

这一小小节开头的各个公式是在行间公式环境内的,若是将它们放入行内公式环境中,
\begin{equation*}\setcounter{MaxMatrixCols}{12}
	\begin{matrix}
		90^{\codegreen{\circ}} & x_{\red{1}} & y^{\codegreen{\prime}} & z^{\color{codegreen} 2} & \mathrm{C}_{\red{n}}^{\codegreen{m}} & \sup_{\red{x \in I}} & \int_{\red{0}}^{\codegreen{1}} & \int_{\red{D}} & \oint_{\red{|z| = 1}} & \prod_{\red{k = 1}}^{\codegreen{n}} & \sum_{\red{1 \leqslant i < j \leqslant m}} & \tensor{T}{\red{\nu_{1} \nu_{2} \cdots \nu_{l}}}{\codegreen{\mu_{\red{1}} \mu_{\red{2}} \cdots \mu_{\red{k}}}}
	\end{matrix}
\end{equation*}
会发现,原本在字符正上、下方的内容移到了右侧的上、下方,原本就在右上、右下的内容相对字符的位置也发生了改变,甚至积分号等等的字符本身就缩小了,这是因为行内公式为了避免产生拥挤而自动作出的调整,可以在环境前用 \lstinline!\displaystyle! 恢复之前的排版效果。公式的字号暂时不宜做深入的讨论。

很自然的想法,如何在符号的左上、左下方增加角标?这里只给出本人认为最合适的解决方法:使用 \verb'mathtools' 宏包给出的
\begin{lstlisting}[numbers=none]
	\prescript{"上标"}{"下标"}{"元素"}
\end{lstlisting}
命令。例如:分子轨道理论解释了氢分子离子$\prescript{1}{1}{\mathrm{H}}_{2}^{+}$的存在,相应的代码为 \lstinline!\prescript{1}{1}! \lstinline!{\mathrm{H}}_{2}^{+}! 。

在张量分析中,上下标不再是$A_{m}^{n}$而是$\tensor{A}{m}{n}$或是$A_{m}^{\phantom{m} n}$,此时最佳的解决方案是利用“幻影”命令 \lstinline!\phantom{}!,这一命令产生一个和参数一样大小的盒子,没有任何内容,起到占位作用。例如:
\begin{lstlisting}[numbers=none]
	A_{m}^{\phantom{m} n}
\end{lstlisting}
输出$A_{m}^{\phantom{m} n}$。但是如果上、下标是连续的、复杂的,代码会变得格外笨重,不如调用 \verb'tensor' 宏包。这里照搬刘海洋书上的例子\cite{LaTeX入门}:
\begin{lstlisting}[numbers=none]
	$M \indices{^a_b^{cd}_e}$ \qquad $\tensor[^a_b^c_d]{M}{^a_b^c_d}$
\end{lstlisting}
输出得到$M_{\phantom{a} b \phantom{c d} e}^{a \phantom{b} c d}$ \qquad ${}^{a \phantom{b} c}_{\phantom{a} b \phantom{c} d} M_{\phantom{a} b \phantom{c} d}^{a \phantom{b} c}$。

化学式中经常出现上、下标,例如硫代硫酸氢根离子 \ce{HS2O3-}。利用数学模式也可以实现这一化学式的输出:
\begin{lstlisting}[numbers=none]
	HS$_{2}$O$_{3}^{-}$
\end{lstlisting}
,效果为HS$_{2}$O$_{3}^{-}$,但是减号的形状不一样,而且代码很复杂。化学宏包 \verb"mhchem" 提供的命令 \lstinline'\ce{}' 可以简化代码。例如:\lstinline'\ce{HS2O3-}'。最后照搬刘海洋书上的反应方程式例子\cite{LaTeX入门}:
\begin{equation}
	\ce{2H2 + O2 ->[\text{燃烧}] 2H2O},
\end{equation}
相应的代码为
\begin{lstlisting}
	\begin{equation}
		\ce{2H2 + O2 ->[\text{"燃烧"}] 2H2O},
	\end{equation}
\end{lstlisting}

补充说明一点:\LaTeX{} 中没有角度符号,刘海洋的处理方法是借用映射复合的二元运算符$\circ$,将其置于上标表示角度。因此代码 \lstinline!90^{\circ}! 输出$90^{\circ}$。

\subsubsection{划线、箭头和花括号}

这里都是指在字符上、下的划线、箭头和花括号。

\lstinline!\overline{}! 和 \lstinline!\underline{}! 命令生成上划线和下划线,例如:表示量子数$m$的上界和下界,即$\underline{m} \leqslant m \leqslant \overline{m}$,相应的代码为 \lstinline!\underline{m} \leqslant m \leqslant \overline{m}!;平面几何中线段表示为$\overline{OP}$,相应的代码为 \lstinline!\overline{OP}!。

上、下箭头的命令及输出效果列于表 \ref{table:上下箭头} 中。
\begin{table}[H]
\centering
\begin{tabular}{cc|cc}
	\toprule
	命令 & 输出效果 & 命令 & 输出效果 \\
	\midrule
	\lstinline!\overrightarrow{OP}! & $\overrightarrow{OP}$ & \lstinline!\underrightarrow{OP}! & $\underrightarrow{OP}$ \\
	\lstinline!\overleftarrow{OP}! & $\overleftarrow{OP}$ & \lstinline!\underleftarrow{OP}! & $\underleftarrow{OP}$ \\
	\lstinline!\overleftrightarrow{OP}! & $\overleftrightarrow{OP}$ & \lstinline!\underleftrightarrow{OP}! & $\underleftrightarrow{OP}$ \\
	\bottomrule
\end{tabular}
\caption{上、下箭头的命令及输出效果}
\label{table:上下箭头}
\end{table}

\lstinline!\overbrace! 和 \lstinline!\underbrace! 命令生成上花括号和下花括号,它们本身可以加上、下标,效果例如:对于本原勾股数$\left( x, y, z \right)$,有
\begin{equation}
	\left( x^{2} z^{2} + y^{4} \right)^{2} = \underbrace{\left( x y \right)^{2} + \left( y z \right)^{2} + \left( x z \right)^{2}}_{\text{3项}}
\end{equation}
相应的代码为
\begin{lstlisting}[numbers=none]
	\left( x^{2} z^{2} + y^{4} \right)^{2} = \underbrace{\left( x y \right)^{2} + \left( y z \right)^{2} + \left( x z \right)^{2}}_{\text{3"项"}}
\end{lstlisting}
,其中在数学模式中用 \lstinline'\text{}' 输入了中文,中文直接在数学模式中是不能正常输入显示的。

\verb'mathtools' 宏包提供了产生上、下方括号的命令:
\begin{lstlisting}
	\overbracket["线宽"]["伸出高度"]{"加括号内容"}
	\underbracket["线宽"]["伸出高度"]{"加括号内容"}
\end{lstlisting}

刘海洋的书\cite{LaTeX入门}中有一道有趣的习题:考虑如何排版下列交错的花括号?
\begin{equation}
	a + \rlap{$\overbrace{\phantom{b + c + d}}^{m}$} b + \underbrace{c + d + e}_{n} + f
\end{equation}

\subsubsection{分式}

最基本的分式(fraction)由
\begin{lstlisting}[numbers=none]
	\frac{"分子内容"}{"分母内容"}
\end{lstlisting}
生成。观察以下的例子:
\begin{align*}
%	\begin{aligned}
		\frac{1}{2} && \text{文本中的$\textstyle \red{\frac{1}{2}}$} && \frac{1}{1 + \red{\frac{1}{2}}} && \frac{\red{\frac{1}{2}}}{1 + \red{\frac{1}{2}}}
%	\end{aligned}
\end{align*}
发现分式在行内公式、分子和分母中的大小是被压缩的。这不是偶然,以后会详细介绍公式的大小控制,包括 \lstinline'\displaystyle' 和 \lstinline'\textstyle'。可以用 \lstinline'\dfrac{}{}' 和 \lstinline'\tfrac{}{}' 生成不受环境影响的行间和行内公式大小的分式。被压缩的分数式最好改为带有括号的横式,例如将$\frac{1}{ab}$改为$1/(ab)$。

设黄金分割比(the golden ratio)为$g$,则有
\begin{equation}
	g = \cfrac{1}{1 + \cfrac{1}{1 + \cdots}},
\end{equation}
这一连分数形式的排版由 \verb'amsmath' 提供的 \lstinline!\cfrac[]{}{}! 完成,其中的可选参数可以为 \lstinline!l!、\lstinline!c! 和 \lstinline!r!,分别设置分子左对齐、居中和右对齐,默认居中。源代码为:
\begin{lstlisting}[numbers=none]
	g = \cfrac{1}{1 + \cfrac{1}{1 + \cdots}},
\end{lstlisting}

二项式系数 \lstinline!\binom{n}{k}! 的用法和分式很像,例如根据二项式定理展开$\left( a + b \right)^{3}$有
\begin{equation}
	\left( a + b \right)^{3} = \binom{3}{0} a^{3} + \binom{3}{1} a^{2} b + \binom{3}{2} a b^{2} + \binom{3}{3} b^{3},
\end{equation}
这个代码就不写出来了。同理有命令 \lstinline'\dbinom{}{}' 和 \lstinline'\tbinom{}{}'。

\subsubsection{根式}

根式由
\begin{lstlisting}[numbers=none]
	\sqrt["根式次数"]{"根式内容"}
\end{lstlisting}
生成。现在仍然不适合涉及根式的精细调整。

\subsection{符号与类型}

数学符号庞杂,但是可以系统地分为普通符号和字母、算子、二元关系符和二元运算符、括号和定界符以及数学标点。不同类别的处理方法是不一样的。

\subsubsection{普通符号和字母}

数字和字母的字体有很多,这里仅仅罗列常用的 \LaTeX{} 提供的数学字体于表 \ref{table:常用数学字体} 中。详细的内容由高然同学负责,他对字体的研究很深刻。
\begin{table}[H]
\centering
\begin{tabular}{cc|l}
	\toprule
	类别 & 命令 & \multicolumn{1}{c}{输出效果} \\
	\midrule
	默认字体 & \lstinline'\mathnormal{}' & $\mathnormal{ABCDHIJKWXYZabcdhijkwxyz0123}$ \\
	意大利体 & \lstinline'\mathit{}' & $\mathit{ABCDHIJKWXYZabcdhijkwxyz0123}$ \\
	罗马体 & \lstinline'\mathrm{}' & $\mathrm{ABCDHIJKWXYZabcdhijkwxyz0123}$ \\
	手写体(花体) & \lstinline'\mathcal{}' & $\mathcal{ABCDHIJKWXYZ}$ \\
	\bottomrule
\end{tabular}
\caption{常用的 \LaTeX{} 提供的数学字体}
\label{table:常用数学字体}
\end{table}

按照排版规范,数学公式变量需要使用意大利体,常数需要使用罗马体。例如:自然常数$\mathrm{e}$、虚数单位$\mathrm{i}$和光速$\mathrm{c}$。

常用的数学字体和字体包如表 \ref{table:常用宏包字体} 所示。
\begin{table}[H]
\centering
\begin{tabular}{ccc|l}
	\toprule
	类别 & 命令 & 宏包 & \multicolumn{1}{c}{输出效果} \\
	\midrule
	黑板粗体 & \lstinline'\mathbb{}' & \verb'amssymb' & $\mathbb{ABCDHIJKWXYZ}$(仅大写字母) \\
	花体 & \lstinline'\mathscr{}' & \verb'mathrsfs' & $\mathscr{ABCDHIJKWXYZ}$(仅大写字母) \\
	哥特体 & \lstinline'\mathfrak{}' & \verb'amssymb' & $\mathfrak{ABCDHIJKWXYZabcdhijkwxyz0123}$ \\
	\bottomrule
\end{tabular}
\caption{常用的数学字体和字体包}
\label{table:常用宏包字体}
\end{table}
\noindent 这里只列举本人使用过很多次的字体。

希腊字母是这一小节的核心,首先是24个希腊字母的大写和小写的输出如表 \ref{table:希腊} 所示。除了$\mathrm{o}$外其他23个小写希腊字母都有定义,而仅有11个大写希腊字母有定义。显然,那些和拉丁字母形状完全一样的希腊字母无需重复定义。
\begin{table}[H]
\centering
\begin{tabular}{cc|cc||cc|cc}
	\toprule
	小写字母 & 命令 & 小写字母 & 命令 & 大写字母 & 命令 & 大写字母 & 命令 \\
	\midrule
	$\alpha$ & \lstinline'\alpha' & $\nu$ & \lstinline'\nu' & $\mathrm{A}$ & \lstinline'\mathrm{A}' & $\mathrm{N}$ & \lstinline'\mathrm{N}' \\
	$\beta$ & \lstinline'\beta' & $\xi$ & \lstinline'\xi' & $\mathrm{B}$ & \lstinline'\mathrm{B}' & $\Xi$ & \lstinline'\Xi' \\
	$\gamma$ & \lstinline'\gamma' & $o$ & \lstinline'o' & $\Gamma$ & \lstinline'\Gamma' & $\mathrm{O}$ & \lstinline'\mathrm{O}' \\
	$\delta$ & \lstinline'\delta' & $\pi$ & \lstinline'\pi' & $\Delta$ & \lstinline'\Delta' & $\Pi$ & \lstinline'\Pi' \\
	$\epsilon$ & \lstinline'\epsilon' & $\rho$ & \lstinline'\rho' & $\mathrm{E}$ & \lstinline'\mathrm{E}' & $\mathrm{P}$ & \lstinline'\mathrm{P}' \\
	$\zeta$ & \lstinline'\zeta' & $\sigma$ & \lstinline'\sigma' & $\mathrm{Z}$ & \lstinline'\mathrm{Z}' & $\Sigma$ & \lstinline'\Sigma' \\
	$\eta$ & \lstinline'\eta' & $\tau$ & \lstinline'\tau' & $\mathrm{H}$ & \lstinline'\mathrm{H}' & $\mathrm{T}$ & \lstinline'\mathrm{T}' \\
	$\theta$ & \lstinline'\theta' & $\upsilon$ & \lstinline'\upsilon' & $\Theta$ & \lstinline'\Theta' & $\Upsilon$ & \lstinline'\Upsilon' \\
	$\iota$ & \lstinline'\iota' & $\phi$ & \lstinline'\phi' & $\mathrm{I}$ & \lstinline'\mathrm{I}' & $\Phi$ & \lstinline'\Phi' \\
	$\kappa$ & \lstinline'\kappa' & $\chi$ & \lstinline'\chi' & $\mathrm{K}$ & \lstinline'\mathrm{K}' & $\mathrm{X}$ & \lstinline'\mathrm{X}' \\
	$\lambda$ & \lstinline'\lambda' & $\psi$ & \lstinline'\psi' & $\Lambda$ & \lstinline'\Lambda' & $\Psi$ & \lstinline'\Psi' \\
	$\mu$ & \lstinline'\mu' & $\omega$ & \lstinline'\omega' & $\mathrm{M}$ & \lstinline'\mathrm{M}' & $\Omega$ & \lstinline'\Omega' \\
	\bottomrule
\end{tabular}
\caption{24个希腊字母的大写和小写的输出}
\label{table:希腊}
\end{table}
\noindent 有8个小写希腊字母存在变体,如表 \ref{table:小写变体} 所示。
\begin{table}[H]
\centering
\begin{tabular}{cc|cc|cc|cc}
	\toprule
	小写字母 & 命令 & 小写字母 & 命令 & 小写字母 & 命令 & 小写字母 & 命令 \\
	\midrule
	$\varepsilon$ & \lstinline'\varepsilon' & $\vartheta$ & \lstinline'\vartheta' & $\varkappa$ & \lstinline'\varkappa' & $\varpi$ & \lstinline'\varpi' \\
	$\varrho$ & \lstinline'\varrho' & $\varsigma$ & \lstinline'\varsigma' & $\varphi$ & \lstinline'\varphi' & $\digamma$ & \lstinline'\digamma' \\
	\bottomrule
\end{tabular}
\caption{8个变体小写希腊字母}
\label{table:小写变体}
\end{table}
\noindent 其中$\varkappa$和$\digamma$是 \AmS{} 符号,需要调用 \verb'amssymb' 等宏包。11个已经定义的大写希腊字母也存在变体,如表 \ref{table:大写变体} 所示,它们和13个拉丁字母的斜体构成了斜体的大写希腊字母全体。
\begin{table}[H]
\centering
\begin{tabular}{cc|cc|cc|cc}
	\toprule
	大写字母 & 命令 & 大写字母 & 命令 & 大写字母 & 命令 & 大写字母 & 命令 \\
	\midrule
	$\varGamma$ & \lstinline'\varGamma' & $\varDelta$ & \lstinline'\varDelta' & $\varTheta$ & \lstinline'\varTheta' & $\varLambda$ & \lstinline'\varLambda' \\
	$\varXi$ & \lstinline'\varXi' & $\varPi$ & \lstinline'\varPi' & $\varSigma$ & \lstinline'\varSigma' & $\varUpsilon$ & \lstinline'\varUpsilon' \\
	$\varPhi$ & \lstinline'\varPhi' & $\varPsi$ & \lstinline'\varPsi' & $\varOmega$ & \lstinline'\varOmega' & ~ & ~ \\
	\bottomrule
\end{tabular}
\caption{11个变体大写希腊字母}
\label{table:大写变体}
\end{table}
\noindent 很自然的考虑,是否可以输出直体的小写希腊字母呢?\verb'upgreek' 宏包应当是最好的解决方案。24个直立小写希腊字母和6个直立变体小写希腊字母如表 \ref{table:小写直体} 所示。
\begin{table}[H]
\centering
\begin{tabular}{cc|cc|cc|cc}
	\toprule
	大写字母 & 命令 & 大写字母 & 命令 & 大写字母 & 命令 & 大写字母 & 命令 \\
	\midrule
	$\upalpha$ & \lstinline'\upalpha' & $\upbeta$ & \lstinline'\upbeta' & $\upgamma$ & \lstinline'\upgamma' & $\updelta$ & \lstinline'\updelta' \\
	$\upepsilon$ & \lstinline'\upepsilon' & $\upzeta$ & \lstinline'\upzeta' & $\upeta$ & \lstinline'\upeta' & $\uptheta$ & \lstinline'\uptheta' \\
	$\upiota$ & \lstinline'\upiota' & $\upkappa$ & \lstinline'\upkappa' & $\uplambda$ & \lstinline'\uplambda' & $\upmu$ & \lstinline'\upmu' \\
	$\upnu$ & \lstinline'\upnu' & $\upxi$ & \lstinline'\upxi' & $\mathrm{o}$ & \lstinline'\mathrm{o}' & $\uppi$ & \lstinline'\uppi' \\
	$\uprho$ & \lstinline'\uprho' & $\upsigma$ & \lstinline'\upsigma' & $\uptau$ & \lstinline'\uptau' & $\upupsilon$ & \lstinline'\upupsilon' \\
	$\upphi$ & \lstinline'\upphi' & $\upchi$ & \lstinline'\upchi' & $\uppsi$ & \lstinline'\uppsi' & $\upomega$ & \lstinline'\upomega' \\
	\bottomrule
\end{tabular}
\begin{tabular}{cc|cc|cc}
	\toprule
	大写字母 & 命令 & 大写字母 & 命令 & 大写字母 & 命令 \\
	\midrule
	$\upvarepsilon$ & \lstinline'\upvarepsilon' & $\upvartheta$ & \lstinline'\upvartheta' & $\upvarpi$ & \lstinline'\upvarpi' \\
	$\upvarrho$ & \lstinline'\upvarrho' & $\upvarsigma$ & \lstinline'\upvarsigma' & $\upvarphi$ & \lstinline'\upvarphi' \\
	\bottomrule
\end{tabular}
\caption{24个直立小写希腊字母和6个直立变体小写希腊字母}
\label{table:小写直体}
\end{table}
\noindent 这样一来,圆周率可以用$\uppi$表示,变分算符用$\updelta$表示,$\mathrm{o}$表示高阶无穷小函数。例如:在$x = 0$附近有Taylor公式
\begin{equation}
	f \left( x \right) = f \left( 0 \right) + f' \left( 0 \right) x + \frac{f'' \left( 0 \right)}{2!} x^{2} + \frac{f''' \left( 0 \right)}{3!} x^{3} + \frac{f^{(4)} \left( 0 \right)}{4!} x^{4} + \mathrm{o} \left( x^{4} \right)
\end{equation}

此外补充一个希伯来字母 \lstinline'\aleph'。例如,连续统假设:$\aleph_{1} = 2^{\aleph_{0}}$。

数学重音(math accents)是给字符加上重音符号。表 \ref{table:accents} 列举了常用的几个数学重音。
\begin{table}[H]
\centering
\begin{tabular}{cc|cc|cc|cc}
	\toprule
	重音 & 命令 & 重音 & 命令 & 重音 & 命令 & 重音 & 命令 \\
	\midrule
	$\dot{q}$ & \lstinline'\dot{}' & $\ddot{q}$ & \lstinline'\ddot{}' & $\dddot{q}$ & \lstinline'\dddot{}' & $\ddddot{q}$ & \lstinline'\ddddot{}' \\
	$\hat{q}$ & \lstinline'\hat{}' & $\vec{q}$ & \lstinline'\vec{}' & $\mathring{q}$ & \lstinline'\mathring{}' & $\tilde{q}$ & \lstinline'\tilde{}' \\
	$\bar{q}$ & \lstinline'\bar{}' & $\widetilde{AB}$ & \lstinline'\widetilde{}' & $\widehat{AB}$ & \lstinline'\widehat{}' & ~ & ~ \\
	\bottomrule
\end{tabular}
\caption{常用的数学重音}
\label{table:accents}
\end{table}
\noindent 其中 \lstinline'\dddot{}' 和 \lstinline'\ddddot{}' 需要调用 \verb'amsmath' 宏包。表 \ref{table:accents} 的第2行内容常用于表示对时间参数或是弧长参数的导数。\lstinline'\hat{}' 一般用于表示量子力学中的一个算符。\lstinline'\mathring{}' 可以和字母A组合得到长度单位$\mathring{\mathrm{A}}$,$1 \,\mathring{\mathrm{A}} = 10^{-10} \,\mathrm{m}$。假设已知偏微分算符$\partial$的命令,就可以排版Euler-Lagrange方程:
\begin{equation}
	\frac{\mathrm{d}}{\mathrm{d} t} \left( \frac{\partial L}{\partial \dot{q}} \right) - \frac{\partial L}{\partial q} = 0
\end{equation}
源代码如下:
\begin{lstlisting}[numbers=none]
	\frac{\mathrm{d}}{\mathrm{d} t} \left( \frac{\partial L}{\partial \dot{q}} \right) - \frac{\partial L}{\partial q} = 0
\end{lstlisting}

还有许多没有照面的符号,诸如约化普朗克常量$\hbar$和表示无穷的$\infty$等等。这些符号称为数学环境的普通符号(ordinary symbols),如表 \ref{table:ordinary} ,它们的字体不会轻易改变,产生的间距与字母相同。
\begin{table}[H]
\centering
\begin{tabular}{cc|cc|cc|cc}
	\toprule
	符号 & 命令 & 符号 & 命令 & 符号 & 命令 & 符号 & 命令 \\
	\midrule
	$\hbar$ & \lstinline'\hbar' & $\ell$ & \lstinline'\ell' & $\Re$ & \lstinline'\Re' & $\Im$ & \lstinline'\Im' \\
	$\infty$ & \lstinline'\infty' & $\prime$ & \lstinline'\prime' & $\emptyset$ & \lstinline'\emptyset' & $\nabla$ & \lstinline'\nabla' \\
	$\partial$ & \lstinline'\partial' & $\angle$ & \lstinline'\angle' & $\triangle$ & \lstinline'\triangle' & $\forall$ & \lstinline'\forall' \\
	$\clubsuit$ & \lstinline'\clubsuit' & $\diamondsuit$ & \lstinline'\diamondsuit' & $\heartsuit$ & \lstinline'\heartsuit' & $\spadesuit$ & \lstinline'\spadesuit' \\
	$\backslash$ & \lstinline'\backslash' & $\hslash$ & \lstinline'\hslash' & $\varnothing$ & \lstinline'\varnothing' & $\square$ & \lstinline'\square' \\
	$\blacksquare$ & \lstinline'\blacksquare' & $\lozenge$ & \lstinline'\lozenge' & $\blacklozenge$ & \lstinline'\blacklozenge' & $\bigstar$ & \lstinline'\bigstar' \\
	$\complement$ & \lstinline'\complement' & $\mho$ & \lstinline'\mho' & ~ & ~ & ~ & ~ \\
	\bottomrule
\end{tabular}
\caption{常用的数学普通符号}
\label{table:ordinary}
\end{table}
\noindent 最后3行除了 \lstinline'\backslash' 以外均为 \AmS{} 符号。\lstinline'\backslash' 再次出现,后面还会有它的踪迹。补充几个在文本模式和数学模式中通用的符号:$\S$和$\dag$,它们的命令分别为 \lstinline'\S' 和 \lstinline'\dag'。$\dag$常用于量子力学中表示Hermite共轭,例如酉算子$\mathcal{U}$满足$\mathcal{U} \mathcal{U}^{\dagger} = \mathcal{U}^{\dagger} \mathcal{U} = 1$。命令 \lstinline'\dagger' 和 \lstinline'\dag' 一样都会在数学模式中输出$\dag$,目前本人唯一知道的区别是 \lstinline'\dagger' 只能用于数学模式。

\subsubsection{算子}

之前和同学讨论,对方简单地认为形如$\sin$和$\sup$等的符号是常量,故使用直体。但是它们不可以视为常量,而成为算子(operator)。算子会在前后自动留出合适的间距,例如:正弦函数$\sin x$。

算子分为
\begin{enumerate}[label=(\arabic*)]

\item 巨算符(large operator),大小随着行间和行内公式变化而变化。例如:
\begin{equation*}
	\begin{aligned}
		\sum && \prod && \bigcup && \bigcap && \bigvee && \bigwedge && \bigsqcup && \bigoplus && \bigotimes && \int && \oint && \iint && \varoiint && \iiint
	\end{aligned}
\end{equation*}
前面几个不可以和它们相似的字母或二元运算符$\Sigma$、$\Pi$、$\cup$、$\cap$、$\vee$、$\wedge$、$\sqcup$、$\oplus$和$\otimes$混用。

\item 文字名称的算子,用直立罗马体排印。例如:$\log$和$\lim$等。

\end{enumerate}

巨算符一览表见表 \ref{table:large} 。
\begin{table}[H]
\centering
\begin{tabular}{cc|cc|cc|cc}
	\toprule
	符号 & 命令 & 符号 & 命令 & 符号 & 命令 & 符号 & 命令 \\
	\midrule
	$\displaystyle \sum \textstyle \sum$ & \lstinline'\sum' & $\displaystyle \prod \textstyle \prod$ & \lstinline'\prod' & $\displaystyle \coprod \textstyle \coprod$ & \lstinline'\coprod' & $\displaystyle \bigcup \textstyle \bigcup$ & \lstinline'\bigcup' \\
	$\displaystyle \bigcap \textstyle \bigcap$ & \lstinline'\bigcap' & $\displaystyle \biguplus \textstyle \biguplus$ & \lstinline'\biguplus' & $\displaystyle \bigsqcup \textstyle \bigsqcup$ & \lstinline'\bigsqcup' & $\displaystyle \bigvee \textstyle \bigvee$ & \lstinline'\bigvee' \\
	$\displaystyle \bigwedge \textstyle \bigwedge$ & \lstinline'\bigwedge' & $\displaystyle \bigodot \textstyle \bigodot$ & \lstinline'\bigodot' & $\displaystyle \bigoplus \textstyle \bigoplus$ & \lstinline'\bigoplus' & $\displaystyle \bigotimes \textstyle \bigotimes$ & \lstinline'\bigotimes' \\
	$\displaystyle \int \textstyle \int$ & \lstinline'\int' & $\displaystyle \oint \textstyle \oint$ & \lstinline'\oint' & $\displaystyle \iint \textstyle \iint$ & \lstinline'\iint' & $\displaystyle \iiint \textstyle \iiint$ & \lstinline'\iiint' \\
	$\displaystyle \iiiint \textstyle \iiiint$ & \lstinline'\iiiint' & $\displaystyle \idotsint \textstyle \idotsint$ & \lstinline'\idotsint' & $\displaystyle \oiint \textstyle \oiint$ & \lstinline'\oiint' & $\displaystyle \varoiint \textstyle \varoiint$ & \lstinline'\varoiint' \\
	\bottomrule
\end{tabular}
\caption{巨算符一览表}
\label{table:large}
\end{table}
\noindent 从 \lstinline'\iint' 开始的命令需要调用 \verb'amsmath' 宏包,从 \lstinline'\oiint' 开始它们需要调用 \verb'esint' 宏包。它们都可以带有上、下标。积分号的上、下标都默认在角标位置,而其他巨算符则在本身上、下方。\lstinline'\limits' 和 \lstinline'\nolimits' 命令可以用来手工调整巨算符上、下标的位置,例如:考察式 \eqref{eq:手工调整巨算符},
\begin{equation}\label{eq:手工调整巨算符}
	\sum\nolimits_{k = 1}^{n} \iint\limits_{x_{k}^{2} + y_{k}^{2} \leqslant 1} r^{2} \,\mathrm{d} S.
\end{equation}
源代码如下:
\begin{lstlisting}[numbers=none]
	\sum\nolimits_{k = 1}^{n} \iint\limits_{x_{k}^{2} + y_{k}^{2} \leqslant 1} r^{2} \,\mathrm{d} S.
\end{lstlisting}

\subsubsection{二元关系符和二元运算符}

\subsubsection{括号和定界符}

\subsubsection{数学标点}

\subsection{\LaTeX{}命令派生法}

考虑这三个命令
\begin{table}[H]
\centering
\begin{tabular}{cccccc}
	\hline
	\lstinline!\subset! & $\subset$ & \lstinline!\subseteq! & $\subseteq$ & \lstinline!\subseteqq! & $\subseteqq$ \\
	\lstinline!\colon! & $\colon$ & \lstinline!\coloneq! & $\coloneq$ & \lstinline!\coloneqq! & $\coloneqq$ \\
	\lstinline!\sim! & $\sim$ & \lstinline!\simeq! & $\simeq$ & \lstinline!\cong! & $\cong$ \\
	\hline
\end{tabular}
\end{table}

关于中文标点符号的使用,在中文论文中也会用到很多英文,如专业词汇和数学式等,在中、英文混排的时候应该如何使用标点呢?胡伟\cite{LaTeX完全学习手册}认为
\begin{quotation}\kaishu
	在这方面并没有统一的规定。……

	……根据中国大百科全书数学卷的排版格式,所有文本段落都使用全角逗号和空心句号,只有在数学式之中的逗号为半角,例如:“$f \left( x, y \right) = 0$。”,公式中的两个变量用半角逗号分隔,而在公式结尾处使用全角空心句号。
\end{quotation}
而刘海洋\cite{LaTeX入门}认为
\begin{quotation}\kaishu
	值得注意的是,显示公式后面如果有标点符号,应该放在数学环境内部,紧接着公式。而且因为数学模式下不能使用汉字,所以一般就使用西文的半角标点。
\end{quotation}
两人的观点有点矛盾。个人认为:在行内公式执行胡伟的主张,在行间公式执行刘海洋的主张更为合适。例如
\begin{quotation}\kaishu
	在$n$维时空中,共有$n$个基矢$\mathrm{d} x^{0}, \mathrm{d} x^{1}, \dots, \mathrm{d} x^{n-1}$,它们也是1形式空间的基矢。2形式空间的坐标基矢记作
\begin{equation}
	\mathrm{d} x^{\mu} \wedge \mathrm{d} x^{\nu} = \mathrm{d} x^{\mu} \otimes \mathrm{d} x^{\nu} - \mathrm{d} x^{\nu} \otimes \mathrm{d} x^{\mu} = \tensor{\delta}{\kappa \lambda}{\mu \nu} \,\mathrm{d} x^{\kappa} \otimes \mathrm{d} x^{\lambda}.
\end{equation}
\end{quotation}

\subsection{自定义命令}

\bibliography{参考文献昆明池}%打印参考文献列表

\end{document}
