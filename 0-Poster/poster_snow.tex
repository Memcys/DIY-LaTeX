%%%% Modefied by memcys@github.com on 2019/09/31
%%%% Based on Better Poster latex template example v1.0 (2019/04/04)
%%%% GNU General Public License v3.0
%%%% Rafael Bailo
%%%% https://github.com/rafaelbailo/betterposter-latex-template
%%%% 
%%%% Original design from Mike Morrison
%%%% https://twitter.com/mikemorrison

\documentclass[a0paper,fleqn]{betterposter}
\usepackage[scheme=plain]{ctex}
\usepackage{tikz}
\usepackage{multirow}
\usepackage{tcolorbox}
\usepackage{multicol}
\usepackage[export]{adjustbox}

\usetikzlibrary{mindmap,calendar,shadows,fadings,backgrounds,
decorations.footprints,shadings,%
decorations.fractals,%
decorations.pathmorphing,%
}

% Please add additional folder addresses for images below
\graphicspath{{./},%{another-directory}
}

\newcommand{\Time}{10:00--10:30 AM, 2019 年 9 月 8 日,周日}
\newcommand{\place}{玉泉图书馆二层会议室}

%%%% Uncomment the following commands to customise the format

% Setting the width of columns
% Left column
\setlength{\leftbarwidth}{0\paperwidth}
% Right column
\setlength{\rightbarwidth}{0.2\paperwidth}

%% Setting the column margins
% Horizontal margin
\setlength{\columnmarginvertical}{0.05\paperheight}
% Vertical margin
\setlength{\columnmarginhorizontal}{0.02\paperheight}
% Horizontal margin for the main column
\setlength{\maincolumnmarginvertical}{0.\paperheight}
% Vertical margin for the main column
\setlength{\maincolumnmarginhorizontal}{0.\paperheight}

%% Changing font sizes
% Text font
\renewcommand{\fontsizestandard}{\fontsize{50}{60} \selectfont}
% Main column font
\renewcommand{\fontsizemain}{\fontsize{68}{80} \selectfont}
% Title font
\renewcommand{\fontsizetitle}{\fontsize{70}{110} \selectfont}
% Author font
%\renewcommand{\fontsizeauthor}{\fontsize{28}{35} \selectfont}
% Section font
\renewcommand{\fontsizesection}{\fontsize{60}{75} \selectfont}

%% Changing font sizes for a specific text segment
% Place the text inside brackets:
% {\fontsize{28}{35} \selectfont \LaTeX}

%% Changing colours
% Background of side columns
% \renewcommand{\columnbackgroundcolor}{theory}
% Font of side columns
% \renewcommand{\columnfontcolor}{imperialblue}
% Background of main column
% \renewcommand{\maincolumnbackgroundcolor}{empirical}
% Font of main column
% \renewcommand{\maincolumnfontcolor}{gray}

\newcommand{\mainbgcolor}{green!50!black!50}
\newcommand{\maincolumnwidth}{\paperwidth - \rightbarwidth - \leftbarwidth}

\renewcommand{\maincolumnbackgroundcolor}{\mainbgcolor}
\renewcommand{\maincolumnfontcolor}{black}
\renewcommand{\columnbackgroundcolor}{imperialblue}
\renewcommand{\columnfontcolor}{white}

\begin{document}	
\betterposter{
%%%%%%%% MAIN COLUMN
\maincolumn{
% TikZ
\begin{tikzpicture}[%
  block/.style={
      rectangle,
      thick,
      fill=white,
      align=center,
  },
  title/.style={
    rectangle,
    very thick,
    fill=white,
    align=center,
  },
]
  % Background
  \coordinate (front) at (0,-.42\paperheight);
  \coordinate (horizon) at (0,-.3\paperheight);
  \coordinate (bottom) at (0,-.5\paperheight);
  \coordinate (sky) at (0,.5\paperheight);
  \coordinate (left) at (-.5\maincolumnwidth,0);
  \coordinate (right) at (.5\maincolumnwidth,0);

  \shade [bottom color=blue!30!black!10,top color=blue!30!black!50]
    ([yshift=-1cm]horizon -|  left) rectangle (sky -| right);
  \shade [bottom color=black!70!green!25,top color=black!70!green!10]
    (front -| left) -- (horizon -| left)
    decorate [decoration={random steps,amplitude=25pt}] { -- (horizon -| right) }
    -- (front -| right) -- cycle;
  \shade [top color=black!70!green!25,bottom color=black!25]
    (front -| left) -- (horizon -| left)
    decorate [decoration={random steps,amplitude=25pt}] { -- (horizon -| right) }
    -- (front -| right) -- cycle;
  \shade [top color=black!70!green!25,bottom color=black!25]
    ([yshift=-5mm-1pt]front -| left) rectangle ([yshift=1pt]front -| right);
  \fill [black!25] (bottom -| left) rectangle ([yshift=-5mm]front -| right);

  % Snow
  \foreach \i in {0.05,0.075,...,3}
    \fill [white,decoration=Koch snowflake,opacity=.73]
          [shift=(horizon),shift={(rand*80,rnd*65)},scale=\i]
          [double copy shadow={opacity=0.2,shadow xshift=-5pt,shadow
            yshift=15*\i pt,fill=blue!5,draw=none}]
      decorate {
        decorate {
          decorate {
            (0,0) -- ++(60:1) -- ++(-60:1) -- cycle
          }
        }
      };

  % Mindmap
  \begin{pgflowlevelscope}{\pgftransformscale{4.5}}
    \begin{scope}[
      xshift=-7.ex,
      yshift=2.5ex,
      mindmap,every node/.style={concept, circular drop shadow={%path fading={circle with fuzzy edge 5 percent}
      shadow xshift=.15ex, shadow yshift=-.2ex}, execute at begin node=\hskip0pt},
      root concept/.append style={
        concept color=black,
        fill=blue!5!white,line width=.16ex,
        text=black,%font=\scshape
        },
      text=white,
      elements/.style={concept color=red!50!black,faded/.style={concept color=red!50!black!50}},
      math/.style={concept color=blue,faded/.style={concept color=blue!50}},
      debug/.style={concept color=orange,faded/.style={concept color=orange!50}},
      tikz/.style={concept color=green!50!black,faded/.style={concept color=green!50!black!50}},
      chinese/.style={concept color=green!50!blue,faded/.style={concept color=green!50!blue!50}},
      ref/.style={concept color=red!50!blue,faded/.style={concept color=red!50!blue!50}},
      grow cyclic,
      level 1/.append style={%level distance=4.5cm,
      sibling angle=60,font=\Large,
      },
      % level 2/.append style={level distance=3cm,sibling angle=45,font=\scriptsize
      %}
      ]
    \node[root concept] (LaTeX) {{\Huge\LaTeX} \\{\LARGE 讨论主题}}% root
      child[math] {node[yshift=-1cm] (Math) {公式定理}
        child {node (AMS) {数学模式、AMS 宏集}}
        child {node(Line) {行内、行间及多行公式} }
        child[faded] {node(Sep) {公式间距 字体控制 定理环境} }
        }
      child[tikz, faded] {node[yshift=-1cm] (TikZ) {Ti\emph{k}Z \\绘图}}
      child[elements] {node[xshift=1cm,yshift=-1cm]  (Elements) {文档元素}
        child {node(FigTab) {表格和图片} }
        child {node(Sec) {章节、目录、标题、边脚注} }
        child {node(List) {列表、引用、摘要环境} }
        child {node(Cross ref) {交叉引用} }
        child[faded] {node(Box) {盒子、浮动体} }
        }
      child[ref] {node[yshift=0cm] (Ref) {文献索引}
        child {node(bib) {bib 条目的获取、自动生成} }
        child {node(biblatex) {biblatex natbib} }
        child[faded] {node(Index) {索引} }
        }
      child[chinese] {node[yshift=0cm] (Chinese) {中文排版}}
      child[debug] {node[yshift=0cm] (Debug) {排错寻助}
        child[faded] {node (Log) {编译日志}}
        child {node (MWE) {Minimal Working Example}}
        child {node (Doc) {宏包文档阅读}}
        child {node (Community) {\TeX 论坛社区群组}}
      };
  \end{scope}
  \end{pgflowlevelscope}

  \LARGE
  \calendar[day list downward,
            month text=\textcolor{blue!80!white}{\%mt}\ \%y0,
            month yshift=1.5em,
            name=cal,
            at={(-.85\textwidth,.5\textheight-3cm)},
            dates=2019-09-01 to 2019-11-last]
    if (Sunday)
      [blue]
    else [black!30]
    if (day of month=1) {
      \node at (0pt,1.5em)[anchor=base west][black] {\huge\tikzmonthtext};
    };

    \node [anchor=west] at (cal-2019-09-08.east) {
    {\begin{tcolorbox}[leftrule=.3cm,width=10cm,
      colback=brown!37!white,colframe=green!50!blue!50]
      \textbf{UCAS \LaTeX er 主题讨论邀请会}
    \end{tcolorbox}}
    };

    \draw [red] (cal-2019-09-08) circle (12pt);

  \node[xshift=-32cm,yshift=-31cm] (QR base) {};

  % QR Code pictures
  \node[xshift=-30cm,] (Invitation) at (QR base) {\includegraphics[width=8cm]{qr-text3.jpg}};
  \node[block,xshift=-10cm,] (QQ) at (QR base) {\includegraphics[width=8cm]{qr-qq}};
  \node[block,xshift=10cm,] (WeChat) at (QR base) {\includegraphics[width=8cm]{qr-wechat}};
  \node[block,xshift=30cm,] (GitHub) at (QR base) {\includegraphics[width=8cm]{qr-url}};

  \node [title,anchor=south] at (Invitation.north) {\Huge 活动邀请函};
  \node [title,anchor=south] at (QQ.north) {\Huge 活动 QQ 群};
  \node [title,anchor=south] at (WeChat.north) {\Huge 活动微信群};
  \node [title,anchor=south] at (GitHub.north) {\Huge 海报源码};

  % Host
  \node [very thick,scale=2,yshift=-3cm] at (QR base) {主办:$\qquad$ 玉泉图书馆 $\qquad$ 信息达人社};
  % \node [very thick,scale=2,yshift=-4cm] at (QR base) {\color{blue!50!black!50}Version: 01/09/2019 Snow A0paper};
\end{tikzpicture}
}{
%%%% Bottom space
% Blank
}
}{
%%%%%%%% LEFT COLUMN
% Blank
}{
%%%%%%%% RIGHT COLUMN
\includegraphics[width=1.35\linewidth,center]{invitation.pdf}

\section{\textcolor{orange}{DIY \LaTeX} 系列活动}
一周一主题,轮流报告%\\
%邀请会分配主题到人


\section{\LaTeX 使用场景}
% \hskip-25pt
% \begin{multicols}{2}
% \begin{itemize}
% \item 论文排版
% \item 课程作业
% \item 简历制作
% \item Beamer
% \item 书籍排版
% \item 建模报告
% \item 海报制作
% \item ……
% \end{itemize}
% \end{multicols}
% \begin{center}
\begin{tabular}{c@{\qquad}c}
论文排版 & 书籍排版\\
课程作业 & 建模报告\\
简历制作 & 海报制作\\
Beamer & ……
\end{tabular}
% \end{center}

\section{邀请会详情}
\begin{tabular}{l@{\quad}p{.75\textwidth}}
对象 & (U)CAS 学生\\ & \textcolor{orange}{\textbf{无视} \LaTeX 基础} \\[1ex]
内容 & “一行代码”竞答\\ & 讨论\textbf{系列活动}详情\\ & 分配主题到人 \\[1ex]
时间 & 10:00--10:45 AM\\ & 9 月 8 日,周日 \\[1ex]
地点 & 玉泉图书馆\\ & 二层会议室 \\[1ex]
建议 & 携带可编译\\ & \LaTeX 的工具 \\[1ex]
福利 & 到现场即可\\ & 领取小礼品 \\
\end{tabular}
}
\end{document}
