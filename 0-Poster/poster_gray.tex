%%%% Modefied by memcys@github.com on 2019/09/30
%%%% Better Poster latex template example v1.0 (2019/04/04)
%%%% GNU General Public License v3.0
%%%% Rafael Bailo
%%%% https://github.com/rafaelbailo/betterposter-latex-template
%%%% 
%%%% Original design from Mike Morrison
%%%% https://twitter.com/mikemorrison

\documentclass[a0paper,fleqn]{betterposter}
\usepackage[scheme=plain]{ctex}
\usepackage{tikz}
\usepackage{multirow}
\usepackage{tcolorbox}
\usepackage{multicol}

\usetikzlibrary{mindmap,calendar,shadows,fadings,backgrounds}

\graphicspath{{./},}

\setCJKmainfont{Noto Sans CJK SC}
% \ctexset{
%     fontset=adobe
% }

%\newcommand{\sub1}{%
%UCAS \LaTeX er 主题讨论邀请会%
%}
%\newcommand{\sub2}{DIY \LaTeX}
\newcommand{\Time}{10:00--10:30 AM, 2019 年 9 月 7 日,周六}
\newcommand{\place}{玉泉图书馆二层会议室}

%%%% Uncomment the following commands to customise the format

% Setting the width of columns
% Left column
\setlength{\leftbarwidth}{0\paperwidth}
% Right column
\setlength{\rightbarwidth}{0.2\paperwidth}

%% Setting the column margins
% Horizontal margin
\setlength{\columnmarginvertical}{0.05\paperheight}
% Vertical margin
\setlength{\columnmarginhorizontal}{0.02\paperheight}
% Horizontal margin for the main column
\setlength{\maincolumnmarginvertical}{0.05\paperheight}
% Vertical margin for the main column
\setlength{\maincolumnmarginhorizontal}{0.\paperheight}

%% Changing font sizes
% Text font
\renewcommand{\fontsizestandard}{\fontsize{58}{65} \selectfont}
% Main column font
\renewcommand{\fontsizemain}{\fontsize{68}{80} \selectfont}
% Title font
\renewcommand{\fontsizetitle}{\fontsize{70}{110} \selectfont}
% Author font
%\renewcommand{\fontsizeauthor}{\fontsize{28}{35} \selectfont}
% Section font
\renewcommand{\fontsizesection}{\fontsize{60}{75} \selectfont}

%% Changing font sizes for a specific text segment
% Place the text inside brackets:
% {\fontsize{28}{35} \selectfont \LaTeX}

%% Changing colours
% Background of side columns
% \renewcommand{\columnbackgroundcolor}{theory}
% Font of side columns
% \renewcommand{\columnfontcolor}{gray}
% Background of main column
% \renewcommand{\maincolumnbackgroundcolor}{empirical}
% Font of main column
% \renewcommand{\maincolumnfontcolor}{gray}

\newcommand{\mainbgcolor}{black!37}

\renewcommand{\maincolumnbackgroundcolor}{\mainbgcolor}
\renewcommand{\maincolumnfontcolor}{black}
\renewcommand{\columnbackgroundcolor}{theory}
\renewcommand{\columnfontcolor}{white}

\begin{document}	
\betterposter{
%%%%%%%% MAIN COLUMN

\maincolumn{
%%%% Main space
% \begin{center}
% \title{%
% %致每一位 UCAS \LaTeX er 的
% %邀请函
% 候选主题
% }
% \end{center}

% \vbox{}

% TikZ
% \pgfdeclarelayer{background}
% \pgfsetlayers{background,main}
\noindent
% \scalebox{2.5}{
\begin{tikzpicture}[
  framed,
  background rectangle/.style={fill=\mainbgcolor},
  % background top/.style={draw=black,line width=3ex},
  block/.style={
      rectangle,
      thick,
      fill=white,
      align=center,
  },
  title/.style={
    rectangle,
    very thick,
    fill=white,
    align=center,
},
]
\begin{pgflowlevelscope}{\pgftransformscale{4.5}}
  \begin{scope}[
    yshift=3ex,
    mindmap,every node/.style={concept, circular drop shadow={%path fading={circle with fuzzy edge 5 percent}
    shadow xshift=.15ex, shadow yshift=-.2ex}, execute at begin node=\hskip0pt},
    root concept/.append style={
      concept color=black,
      fill=black!18!white,line width=.16ex,
      text=black,%font=\scshape
      },
    text=white,
    elements/.style={concept color=red!50!black,faded/.style={concept color=red!50!black!50}},
    math/.style={concept color=blue,faded/.style={concept color=blue!50}},
    debug/.style={concept color=orange,faded/.style={concept color=orange!63}},
    tikz/.style={concept color=green!50!black,faded/.style={concept color=green!50!black!50}},
    chinese/.style={concept color=green!50!blue,faded/.style={concept color=green!50!blue!50}},
    ref/.style={concept color=red!50!blue,faded/.style={concept color=red!50!blue!50}},
    grow cyclic,
    level 1/.append style={%level distance=4.5cm,
    sibling angle=60,%font=\scshape
    },
    % level 2/.append style={level distance=3cm,sibling angle=45,font=\scriptsize
    %}
    ]
  \node[root concept] (LaTeX) {{\Huge\LaTeX} \\{\Large 讨论主题}}% root
    child[math] {node[yshift=-1cm] (Math) {公式定理}
      child {node (AMS) {数学模式、AMS 宏集}}
      child {node(Line) {行内、行间及多行公式} }
      child[faded] {node(Sep) {公式间距、字体控制、定理环境} }
      %child {node(Key Problems) {KeyProblems} }
      }
    child[tikz, faded] {node[yshift=-1cm] (TikZ) {Ti\emph{k}Z \\绘图}}
    child[elements] {node[xshift=1cm,yshift=-1cm]  (Elements) {文档元素}
      child {node(FigTab) {表格和图片} }
      child {node(Sec) {章节、目录、标题、边脚注} }
      child {node(List) {列表、引用、摘要环境} }
      child {node(Cross ref) {交叉引用} }
      child[faded] {node(Box) {盒子、浮动体} }
      %child {node(Programming in Logic) {Programmingin Logic} }
      }
    child[ref] {node[yshift=0cm] (Ref) {文献索引}
      child {node(bib) {bib 条目的获取、自动生成} }
      child {node(biblatex) {biblatex/natbib} }
      child[faded] {node(Index) {索引} }
      %child[faded] {node(Describing Complexity) {DescribingComplexity} }
      }
    child[chinese] {node[yshift=0cm] (Chinese) {中文排版}}
    child[debug] {node[yshift=0cm] (Debug) {简单排错和寻求帮助}
      child[faded] {node (Log) {编译日志}}
      child {node (MWE) {Minimal Working Example}}
      child {node (Doc) {宏包文档阅读}}
      child {node (Community) {\LaTeX 论坛社区群组}}
    };
\end{scope}

\end{pgflowlevelscope}

  \LARGE
  \calendar[day list downward,
            month text=\textcolor{blue!80!white}{\%mt}\ \%y0,
            month yshift=1.5em,
            name=cal,
            at={(-.5\textwidth-5mm,.5\textheight-1cm)},
            dates=2019-09-01 to 2019-11-last]
    % \node at (0pt,1.5em)[anchor=base west][black] {\huge\tikzmonthtext}
    if (Sunday)
      [red!80!white]
    % if (at least=2019-09-02, Sunday)
    %   [black!25]
    else [black!15]
    if (day of month=1) {
      \node at (0pt,1.5em)[anchor=base west][black] {\huge\tikzmonthtext};
    };

    \node [anchor=west] at (cal-2019-09-08.east) {%\textcolor{blue}
    {\begin{tcolorbox}[leftrule=.3cm,width=10cm,
      colback=brown!37!white,colframe=green!50!blue!50]
      \textbf{UCAS \LaTeX er 主题讨论邀请会}
    \end{tcolorbox}}
    };

    \draw [red] (cal-2019-09-08) circle (12pt);

  \node[xshift=0cm,yshift=-25cm] (QR base) {};

    % QR Code pictures
  \node[block,xshift=-30cm,] (Invitation) at (QR base) {\includegraphics{qr-text3}};
  \node[block,xshift=-10cm,] (QQ) at (QR base) {\includegraphics[width=6.3cm]{qr-qq}};
  \node[block,xshift=10cm,] (WeChat) at (QR base) {\includegraphics[width=6.3cm]{qr-wechat}};
  \node[block,xshift=30cm,] (GitHub) at (QR base) {\includegraphics{qr-url}};

  \node [title,anchor=south] at (Invitation.north) {活动邀请函};
  \node [title,anchor=south] at (QQ.north) {活动 QQ 群};
  \node [title,anchor=south] at (WeChat.north) {活动微信群};
  \node [title,anchor=south] at (GitHub.north) {海报源码};

  % Host
  \node [very thick,scale=2,yshift=-3cm] at (QR base) {主办:$\qquad$ 玉泉图书馆 $\qquad$ 信息达人社};
  \node [very thick,scale=2,yshift=-4cm] at (QR base) {\color{blue}Version: 31/09/2019 Gray A0paper};

\end{tikzpicture}
% }


% \textbf{UCAS \LaTeX er},\\[.1\baselineskip]

% 你好!我们诚挚地邀请你参加本次%“\textbf{UCAS \LaTeX er 主题讨论
% 邀请会%}”
% 及后续 ``\textbf{DIY \LaTeX}'' 的系列活动。\\[.1\baselineskip]

% 邀请会将决定后续系列活动的%时间、地点、主题、报告人等
% 执行细节。后续活动拟采取\textbf{轮流报告机制}%,
% %即分配(给定或自选)主题到多个参与者,每次活动一个报告主题的机制
% 。\\[.1\baselineskip]

% %我们预期将根据各主题报告整理出一份面向国科大学生的 \LaTeX 入门或进阶使用的文档 ``\textbf{UCAS \LaTeX User Guide}'', 并在网络上公开。\\[.1\baselineskip]

% 此次邀请对象包括但不限于——%日常问候
% \LaTeX \textbf{高手}、%,初出茅庐的
% \textbf{新手},%,抑或无 \LaTeX 写作经验的\textbf{远观者}。%我们都期待您的参与!
% %只要你对 
% 及\textbf{无使用经验者}。
% ``\textbf{DIY \LaTeX}'', % 感兴趣,我们都欢迎你的到来!
% %需要你的热心参与!\\[.1\baselineskip]
% 期待每一个你的参与!

% \vbox{}

% % \begin{itemize}
% % \item 本邀请会
% \begin{itemize}
% \item 时间:\Time
% \item 地点:\place
% \end{itemize}
% \item 后续系列活动时间地点待议
%\begin{itemize}
%\item 时间:后续每周六 10:00--10:30 AM (待议)
%\item 地点:玉泉图书馆二层会议室(待议)
%\end{itemize}
% \end{itemize}

}{
%%%% Bottom space

%% QR code
% \qrcode{img/qrcode-eps-converted-to}{img/smartphoneWhite}{
% \textbf{Take a picture} to
% \\download the full paper
% }
% Smartphone icon
% Author: Freepik
% Retrieved from: https://www.flaticon.com/free-icon/smartphone_65680

%% Compact QR code (comment the previous command and uncomment this one to switch)
% \compactqrcode{img/qrcode-eps-converted-to}{
% \textbf{Take a picture} to
% \\download the full paper
% % QR code for T
% }TeX/
% ~
% \includegraphics{TeX/rotated-triangle.pdf}
% ~
% \includegraphics[page=3]{TeX/rotated-polygons.pdf}

}
}{
%%%%%%%% LEFT COLUMN
% Blank
}{
%%%%%%%% RIGHT COLUMN

\title{%
\begin{tabular}{rp{.35\linewidth}}
UCAS & \multirow{3}{*}{\includegraphics[width=.6\textwidth]{invitation.pdf}} \\%[.1\baselineskip]
\LaTeX er\\[.2\baselineskip]
主题讨论 \\ %\\[\baselineskip]
% · 
% \includegraphics[width=.65\ccwd]{TeX/circular_glow.pdf}
% 邀请会
\end{tabular}
}
%\author{memcys@github.com}
% \institution{玉泉图书馆二层会议室}
% \date{2019-09-07}

% \section{候补母主题}
% %Here is an itemised list:
% \begin{itemize}
% \item 中文排版
% \item 公式定理
% \item 文档元素
% \item 文献索引
% \item TikZ 绘图
% \item 宏包文档阅读
% \item 简单排错和提问
% \item ……
% \end{itemize}
%更多母主题及相应子主题将从邀请会得出。

\section{\LaTeX 使用场景}
\begin{multicols}{2}
\begin{itemize}
% \item arXiv/期刊/会议论文排版
\item %毕业
论文排版%(图片可放附件 模板使用说明.pdf 的首页截图)
\item Beamer %(图片可放如 \url{https://hartwork.org/beamer-theme-matrix/all/beamer-default-Copenhagen-0.png})
\item 课程作业
\item 建模报告
\item 书籍排版
\item 简历制作
\item 海报制作
\item ……
\end{itemize}
\end{multicols}

\section{邀请会详情}
% \begin{flushleft}

\begin{tabular}{l@{\quad}p{.6\textwidth}}
时间 & 10:00--10:45\\ & 9 月 8 日 \\ & 周日 \\[1ex]
地点 & 玉泉图书馆\\ & 二层会议室 \\[1ex]
建议 & 携带可编译\\ & \LaTeX 的工具 \\[1ex]
提示 & 到场即可\\ & 领取小礼品 \\[5ex] 
% \end{tabular}

% \vbox{}
% \\\hline
% \begin{tabular}{l@{\quad}p{.6\textwidth}}
主办 & 信息达人社\\
& 玉泉图书馆
\end{tabular}
% \end{flushleft}
% Some cute ducklings:
% \begin{center}
% % Picture of ducklings
% % Author: Magda Ehlers, https://www.pexels.com/@magda-ehlers-pexels
% % Retrieved from: https://www.pexels.com/photo/selective-focus-photo-of-flock-of-ducklings-perching-on-gray-concrete-pavement-1300355/
% \includegraphics[width=\textwidth]{img/ducklings}
% \end{center}
}
\end{document}
